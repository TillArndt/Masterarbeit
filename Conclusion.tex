\chapter{Conclusion}

This analysis presents an overview of the ongoing investigation of \ttgamma events in the semi-muonic channel. \\
In Chapter \ref{sec_simu} a new simulation strategy for \ttgamma events is introduced. This new generation approach closely represents the measurement, as it includes all the measured final state particles. The computing time is reduced to one tenth through factorization compared to the non factorized approach. Its validity is verified by comparing the factorized to the non-factorized method.
Two LO generators, MADGRAPH and WHIZARD, are compared to each other. They agree with each other on calculated cross sections as well as on the kinematic properties like $\pt (\gamma)$, $\Delta \mathrm{R} (\gamma,b)$, $\eta (\gamma)$, etc. . Consequently, they can be used interchangeably. \\
\\
In Chapter \ref{sec_ano} a benchmark study for the measurement of the magnetic moment of the top-quark is presented. A non-zero magnetic moment leads to a top-quark radius larger than zero.  The simulation strategy described above is used to generate several benchmark scenarios with WHIZARD. The sensitivity lies in the region of $d_V^{\gamma} \leq 2$ corresponding to a top-quark radius of $R_t=\SI{5d-3}{\femto \meter}$ or \SI{0.5}{\percent} of the proton radius.\\
\\
In Chapter \ref{sec_ttg} the cross section of the $pp \to \ttgamma$ process at $\sqrt{s} = \SI{8}{\tera \electronvolt}$ is measured using \SI{19.6}{\per \femto \barn} of CMS data. First a pre-selection of \ttbar events is implemented. After the pre-selection, $N^{presel} =256665$ events remain in data with a purity of $\pi_{\ttbar} = \SI{84.3}{\percent}$. In the next step, events with a photon candidate are selected. Based on the loose CMS selection, $N^{sel} = 7836$ events remain with an efficiency of $\epsilon_\gamma = \SI{62.9}{\percent}$ and a purity of $\pi_{\ttgamma} = \SI{66.7}{\percent}$. The number of real photons is extracted with a data-driven template fit and $N_{real} = 1894 \pm 76$ \ttgamma events are found in data. The template fit is validated with multiple closure tests. Inconsistencies found in the impact of different pile-up simulations are propagated to the final result as systematic uncertainties. This uncertainty, related to the out-of-time pile-up, belongs to the dominant systematic uncertainties along with those introduced by the generator and the signal cross section. The final result is (see Equation \ref{eq_con_ttg}):
\begin{equation}
\sigma_{\ttgamma} \; =\; 2.6\;\pm \;0.1\mathrm{ (stat.)} \;\pm \;0.5\mathrm{ (syst.)}\; \SI{}{\pico \barn} \; .
\label{eq_con_ttg}
\end{equation}

This is an improvement of the previous CMS result of $\sigma_{\ttgamma}^{\mathrm{CMS}} = \SI[parse-numbers=false]{2.4 \pm 0.2(stat.) \pm 0.6(syst.)}{\pico \barn}$ . The result also agrees with the theory prediction of $\sigma_{\ttgamma} = \SI{1.8 \pm 0.5}{\pico \barn}$.\\