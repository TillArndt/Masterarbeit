\documentclass[11pt,twoside,a4paper]{book}
\usepackage{amsmath}
\usepackage{color}
\usepackage{upgreek}
\usepackage{hyperref}
\usepackage{a4wide}
\usepackage{multirow}
\usepackage{latexsym}
\usepackage{graphicx}
\usepackage[english]{babel}
\usepackage[babel,english=american]{csquotes}
\usepackage[T1]{fontenc}
\usepackage[hang]{caption}
\usepackage{sparticles}
\usepackage{subfigure}
%\usepackage{icomma} % kein Leerraum nach Komma als Dezimaltrennzeichen,kann Probleme mit dcolumn machen
\usepackage{feynmf}
\usepackage{lscape}
\usepackage[separate-uncertainty,multi-part-units = single,per-mode = reciprocal]{siunitx}

\bibliographystyle{plain}

\definecolor{darkblue}{rgb}{0,0,0.5}

\definecolor{MyColour3}{rgb}{0.0,1.0,0.0} %

\definecolor{MyColour38}{rgb}{1.0,1.0,0.0} %
\definecolor{MyColour94}{rgb}{1.0,0.0,0.0}
\definecolor{MyColour102}{rgb}{0.0,0.0,1.0}
\definecolor{black}{rgb}{0.0,0.0,0.0}
\newcommand{\legend}{\begin{tabular}{clclcl}
 \colorbox{MyColour3}{\textcolor{MyColour3}{\phantom{M}}}     &  \ttb 
& \colorbox{MyColour94}{\textcolor{MyColour94}{\phantom{M}}}   &
\ttb~other &
\colorbox{MyColour102}{\textcolor{MyColour102}{\phantom{M}}} & Single
$t$, $t$-Ch \\[2mm]
 \colorbox{MyColour38}{\textcolor{MyColour38}{\phantom{M}}}   &  Single
$t$, $tW$-Ch &  \colorbox{black}{\textcolor{MyColour38}{\phantom{M}}} 
&  \multicolumn{2}{c}{$W \rightarrow \mu \nu$}    &
\end{tabular} \\}

\hypersetup{colorlinks=true, linkcolor=darkblue, urlcolor=darkblue,
citecolor=darkblue}
\newcommand{\abb}[1]{\hyperref[#1]{Abbildung~\ref*{#1}}}
\newcommand{\tab}[1]{\hyperref[#1]{Tabelle~\ref*{#1}}}
\newcommand{\form}[1]{\hyperref[#1]{Gleichung~\ref*{#1}}}
\newcommand{\eq}[1]{\hyperref[#1]{Gleichung~\ref*{#1}}}
\newcommand{\kap}[1]{\hyperref[#1]{Kapitel~\ref*{#1}}}
\newcommand{\ttb}[0]{$t\overline{t}$}
\newcommand{\hme}[0]{$^{-1}$}
\newcommand{\ttgamma}[0]{\ensuremath{\mathrm{t}\overline{\mathrm{t}} \gamma} }
\newcommand{\ttbar}[0]{\ensuremath{\mathrm{t}\overline{\mathrm{t}}} }
\newcommand{\todo}[1]{\textbf{TODO: #1}}
\newcommand{\pt}{\ensuremath{p_{\mathrm{T}}} }
\newcommand{\PTm}{\ensuremath{{p}_\mathrm{T}\hspace{-1.02em}/}\xspace}
\newcommand{\PTslash}{\ensuremath{{p}_\mathrm{T}\hspace{-1.02em}/}\xspace}
\newcommand{\ETm}[0]{\ensuremath{E_{\mathrm{T}}^{\mathrm{miss}}} }

\setcounter{tocdepth}{2}
\setlength{\parindent}{0in}

\makeatletter
\DeclareRobustCommand*{\bfseries}{%
  \not@math@alphabet\bfseries\mathbf
  \fontseries\bfdefault\selectfont
  \boldmath
}
\makeatother
%\setlength{\parindent}{0em}
\begin{document}
\pagenumbering{roman}
\begin{titlepage}
\thispagestyle{empty}
\begin{center}
   ~ \\
   \vspace{0.2cm}
      {\Huge\bf{Probing the top-quark photon vertex with the CMS detector}\\}
   \vspace{2.0cm}
     von \\
   \vspace{0.5cm}
      {\Large Till Arndt} \\
   \vspace{1.5cm}
      {\Large Masterarbeit in Physik}\\
   \vspace{1.0cm}
     vorgelegt der \\
   \vspace{0.2cm}
      {\Large Fakult\"at f\"ur Mathematik, Informatik und
Naturwissenschaften\\}
      {\Large der Rheinisch-Westf\"alischen Technischen Hochschule
Aachen \\}
   \vspace{1.3cm}
     im \\
   \vspace{0.2cm}
      {\Large Juni 2010}\\
   \vspace{1.5cm}
      angefertigt im \\
   \vspace{0.3cm}
      {\Large III. Physikalischen Institut B \\}
   \vspace{0.3cm}
      bei\\
   \vspace{0.3cm}
      {\Large Prof.~Dr.~A.~Stahl \\}

\end{center}
\end{titlepage}

\tableofcontents
\newpage
\pagenumbering{arabic}
\chapter{Introduction}
\section{Introduction}


adsgasdf


\chapter{The LHC and the CMS detector}

This chapter shortly describes the Large Hadron Collider. In the second part the CMS detector and its most important subsystems are explained.

\section{ The Large Hadron Collider}

The Large Hadron Collider (LHC) \cite{LHCTDR} is located in a tunnel beneath the French Swiss border region near Geneva. The tunnel lies about \SI{100}{\meter} below the ground with a cicumference of \SI{26.7}{ \kilo \meter}. The LHC uses 1232 superconducting magnets to produce a magnetic field with the maximal strength with a of \SI{8.3}{\tesla}.\\
The LHC can accelerate protons as well as heavy ions. For the proton-proton collision two counter rotating beams of protons are accelerated and then brought together at the four interaction points. According to design the proton collisions are provided with a center of mass energy of $\sqrt{s} = \SI{14 }{\tera \electronvolt}$ and a peak luminosity of $\mathcal{L} = \SI{d34}{\per \centi \meter \tothe{2} \per \second}$. \\
The accelerator complex is shown in Figure \ref{fig_det_accel}. In order to reach high energies protons are preaccelerated by the Linear Accelerator 2 (Linac2), the Proton Synchrotron Booster (PSB), the Proton Synchrotron (PS) and finally by the Super Proton Synchrotron (PSP). Through the preacceleration the protons reach a energy of $\SI{450}{\giga \electronvolt}$ for injection into the LHC. \\


\begin{figure}[ht]
    \centering
    \includegraphics[width = 0.7\textwidth] {Bilder/LHC_accel}
  \caption{The Cern accelerator complex. \cite{Lefevre:1165534} The pre accelerators and the four main LHC detectors are shown. Other experiments participate as well.}
  \label{fig_det_accel}
\end{figure}

Four major detectors are located around the LHC ring. \\
CMS \cite{cmsTDR1} and ATLAS \cite{AtlasTDR} are multipurpose detectors. They are designed to cover a large range of physics topics and particles. Additionally, each should be able to corroborate or refute the results of the other experiment allowing for independent validation of these results.\\
ALICE \cite{AliceTDR} is a detector specialized for heavy ion physics. One of the central subjects is the investigation of quark-gluon plasma, possibly created in high energy heavy ion collisions. ALICE is able to record and reconstruct events with a very high particle multiplicity (about 50,000 particles per event). \\
LHCb \cite{LHCbTDR} is only measuring particles on one side of the interaction point. The detector is build for b-physics concentrrating on precision measurements of the standard model for example in the CP-violating sector.\\
LHCf \cite{LHCfTDR} and TOTEM \cite{TOTEMTDR} are situated in the forward region of of ATLAS and CMS respectively. The TOTEM experiment is focused on the proton substructure. LHCf is concerned with development of hadronic showers and the associated phenomenological methods. \\
In 2012 the beam energy was set to \SI{4}{\tera \electronvolt} ($\sqrt{s} = \SI{14}{\tera \electronvolt}$). CMS recorded a luminosity of $\mathcal{L} = \SI{21.79}{\per \femto \barn}$ of the \SI{23.30}{\per \femto \barn} provided by the LHC. \\

\begin{figure}[ht]
  \subfigure[]{
    \includegraphics[width = 0.55\textwidth] {Bilder/lumi_cms_2012}
      }
    \subfigure[]{
    \includegraphics[width = 0.49\textwidth] {Bilder/lumi_lhc_2012}
  }
  \caption{(a): Luminosity delivered by the LHC and recorded by CMS in 2012.\cite{lumi_cms} (b): Luminosity delivered to the four major experiments at the LHC in 2012.\cite{lumi_lhc}}
  \label{fig_det_lumi}
\end{figure}

After the current shutdown of LHC ends in 2015 the center of mass energy will be raised to $\sqrt{s} = \SI{13}{\tera \electronvolt}$.

\section{The Compact Muon Solenoid}

\begin{figure}[ht]
    \includegraphics[width = 0.75\textwidth] {Bilder/cms_complete_labelled}
      
  \caption{Vertical section of the whole cms detector. \cite{CMS_Draw}}
  \label{fig_cms_draw}
\end{figure}
\chapter{Production of Simulated Events}

After discussing the general methods for simulating the dectector response in the CMS collaboration in Section \textbf{TODO: Link}, this chapter will deal with the specific simulation of the signal process applied in this analysis. 
\section{Concept}
\label{ch_simu_con}
In order to simulate \ttgamma events, several strategies are possible. They are shown in Figure \todo{Heiners Plot, plus citation}.\\
The $2 \to 3$ process takes the least time to generate and considers initial state radiation, but it does not take into account any interferences or photon radiation from secondary particles.
This process is later used to compare generators, but otherwise it is not sufficiently accurate for this analysis.\\
The $2 \to 5$ process is a compromise between the time needed for generation and accuracy. Interferences and photon radiation from the top-quark decay are considered up to the W-boson and b-quark. This method has been used in \todo{Cite Heiner}.\\
For a complete simulation of a \ttgamma events the $2 \to 7$ process includes all interference effects and radiation from secondary particles. Idealy it should be used to simulate \ttgamma events, but computation time severly limits its usefulness. \\
In this analysis a factorized $2 \to 7$ approach is used, generating events by combining the processes shown in Equation \ref{eq_simu_fac}. 

\begin{equation}
\begin{split}
& \mathrm{pp}\to \ttgamma,  \mathrm{t} \to \mathrm{bxx},  \overline{\mathrm{t}} \to \overline{\mathrm{b}} \mathrm{xx} \\
& \mathrm{pp}\to \mathrm{t} \overline{\mathrm{t}},  \mathrm{t} \to \mathrm{bxx} \gamma,  \overline{\mathrm{t}} \to \overline{\mathrm{b}} \mathrm{xx} \\
& \mathrm{pp}\to \mathrm{t} \overline{\mathrm{t}},  \mathrm{t} \to \mathrm{bxx},  \overline{\mathrm{t}} \to \overline{\mathrm{b}} \mathrm{xx} \gamma \\
&  x\ = \ \mathrm{u}, \mathrm{d}, \mathrm{s}, \mathrm{c}, \overline{\mathrm{u}}, \overline{\mathrm{d}},  \overline{\mathrm{s}},  \overline{\mathrm{c}}, \mathrm{e}^{\pm}, \mu^{\pm}, \tau^{\pm}, \nu_{\mathrm{e},\mu,\tau}, \overline{\nu}_{\mathrm{e},\mu,\tau} 
\end{split}
\label{eq_simu_fac}
\end{equation}

This approach includes photon radiation from secondary particles and initial state particles, but it only includes interference effects from the initial state particles. \\
The factorized $2 \to 7$ approach should lead to a description which is considerably closer to the "real" \todo{Rephrase} measurement by allowing the matrix element generation of all final state particles.\\
To validate this approach the result from two different generators, MADGRAPH and WHIZARD will be compared in Section \ref{ch_simu_comp}. The results should be compatible. \\
The leading order cross section calculated by the matrix element generators is not reliable, therefore a correction factor should be used in an analysis. This correction factor is difficult to calculate, so this is done by an expert theoretical physicist. \\ 
\todo{Some reason why this works}

\section{Generator Comparison}
\label{ch_simu_comp}

In this section the two leading order matrix element generators MADGRAPH and WHIZARD are compared \todo{Citations} for the $pp \to \ttgamma$ process. MADGRAPH is widely used within the CMS collaboration. WHIZARD has been used in a previous \ttgamma analysis \todo{cite Heiner} and it allows for the test of models with a non Standard Model coupling between top quark and photon. Both are used to generate the matrix element of a given process and produce events in the LHEF format \todo{Link or Citation}. Further effects like hadronization are not considered, as other algorithms like PYTHIA \todo{Citation} are used for this purpose within CMS (see Chapter \todo{Link} for a description of event simulation within CMS). In this section only the output of MADGRAPH and WHIZARD are compared, without any further simulation. 

\subsection{The $2 \to 3$ Process}

As a starting point the $2 \to 3$ process is investigated (see Equation \ref{eq_simu_2to3}). MADGRAPH and WHIZARD are both used two generate events and several kinematic distributions are compared. There should generally be a high degree of conformance.

\begin{equation}
\mathrm{pp} \to \ttgamma
\label{eq_simu_2to3}
\end{equation}

The integral of the matrix element diverges if massless particles are emitted with very low $P_T$ (infrared divergence) or collineary to the mother particle (colinear divergence). In order to make the integral safe to compute kinematic cuts are used to restrict the phase space for photons and jets. Additionally, further cuts are used to speed up the generation process by narrowing the phase space. These cuts have to be below the selection cuts that are used later in the analysis \todo{Link}. Otherwise the final result would depend on the value of the cuts. \\
The kinematic cuts and the technical parameters used for the generation are shown in Table \ref{tab_simu_2to3}. 

\begin{table}[ht]
\centering
    \caption{Technical parameters used for the generation of \ttgamma events (according to\\ Equation \ref{eq_simu_2to3}. The scales are fixed. The parameters are the same for both generators.}
    \begin{tabular}{| l | l |}

    \hline
    Renormalization scale & $ \SI{174.3}{\giga \electronvolt} $ \\
    \hline
    Factorisation scale & $ \SI{174.3}{\giga \electronvolt} $ \\
    \hline
    PDF & cteq6l1 \todo{Citation} \\
    \hline
    \multicolumn{2}{|c|}{Cuts} \\
    \hline
    $\pt(\gamma)$ & $\SI{<13}{\giga \electronvolt}$ \\
    \hline
    $| \eta |(\gamma)$ & $\num{<3}$ \\
    \hline
    $|\eta |(\mathrm{t})$ & $\num{<5}$ \\
    \hline
    \end{tabular}
     \label{tab_simu_2to3}
\end{table}

The calculated cross sections are found in Equation \ref{eq_simu_2to3_cross}. The errors are those given by the leading order calculation basically depending on how much computing time is invested into the calculation. They do not take into account any next-to leading order effects, therefore they are a lot lower than the errors on a theoretical cross section calculation normally found in particle physics. \\
The difference between the generators of $\SI{0.7 }{\percent}$ or $ 1.6\; \sigma$ are consequently not considered significant.

\begin{equation}
\mathrm{WHIZARD: }\; \sigma_{\ttgamma} = \SI{731 \pm 3}{\femto \barn} \hspace{1.cm} \mathrm{MADGRAPH: }\;  \sigma_{\ttgamma} = \SI{735.8 \pm 0.3}{\femto \barn}
\label{eq_simu_2to3_cross}
\end{equation}

Additionally to the calculated cross sections, the events generated by MADGRAPH and WHIZARD should be compared. This is done here by looking  at several kinematic distribution and comparing them to each other. Agreement over a wide range of phase space is expected. \\
The observables shown in Figures \ref{fig_simu_comp_2to3_1} and \ref{fig_simu_comp_2to3_2} agree for MADGRAPH and WHIZARD over a wide range.\\
This shows that MADGRAPH and WHIZARD are compatible for a simple $2 \to 3$ process, but more complex processes still need to be evaluated.

\begin{figure}
  \subfigure[$\pt(\mathrm{t})$]{
    \includegraphics[width = 0.67\textwidth] {Bilder/tquarkPt_Comp_2to3}
  }
  \\
    \subfigure[$\pt(\overline{\mathrm{t}})$]{
    \includegraphics[width = 0.67\textwidth] {Bilder/tquarkPt_Comp_2to3}
  }
      \subfigure[$\pt(\gamma)$ \todo{Caption}]{
    \includegraphics[width = 0.67\textwidth] {Bilder/photonEt_Comp_2to3}
  }
  \caption{Comparison between the MADGRAPH and WHIZARD generators. All distributions are normalized to unity.}
  \label{fig_simu_comp_2to3_1}
\end{figure}

\begin{figure}
  \subfigure[$\pt(\mathrm{t})$]{
    \includegraphics[width = 0.67\textwidth] {Bilder/photonEta_Comp_2to3}
  }
  \\
    \subfigure[$\pt(\overline{\mathrm{t}})$]{
    \includegraphics[width = 0.67\textwidth] {Bilder/tquarkEta_Comp_2to3}
  }
      \subfigure[$\pt(\gamma)$ \todo{Caption}]{
    \includegraphics[width = 0.67\textwidth] {Bilder/dRphotonT_Comp_2to3}
  }
  \caption{Comparison between the MADGRAPH and WHIZARD generators. All distributions are normalized to unity.}
  \label{fig_simu_comp_2to3_2}
\end{figure}


\subsection{The $2 \to 5$ process}

The $2 \to 5$ process has previously been used in a \ttgamma analysis \todo{cite}. Therefore, it is also investigated in this study. Equation \todo{ref} shows the generated reaction. \\

\begin{equation}
\mathrm{pp} \to \mathrm{W}^+ \mathrm{W}^- \mathrm{b} \overline{\mathrm{b}} \gamma
\label{eq_simu_2to5}
\end{equation}

The cuts and scales used for the generation are shown in Table \ref{tab_simu_2to5}. The renormalization and factorization scales are not the same for both generators. This is unfortunate, but due to time constraints a sample generated for \todo{cite Heiner} was used. In this WHIZARD sample the scale was $\SI{172.5}{\giga \electronvolt} + \pt(\gamma)$, which is not possible in MADGRAPH. As it would have taken a lot of time to re-generate an appropriate amount of events, this difference could not be mitigated. \todo{maybe rephrase}\\ 

\begin{table}[ht]
\centering
    \caption{Technical parameters used for the generation of \ttgamma events (according to\\ Equation \ref{eq_simu_2to5}. The scales are fixed in MADGRAPH, in WHIZARD the \pt of the photon is added. The parameters are the same for both generators.}
    \begin{tabular}{| l | l |}

    \hline
    Renormalization scale & $ \SI{172.5}{\giga \electronvolt} $ \\
    \hline
    Factorisation scale & $ \SI{172.5}{\giga \electronvolt} $ \\
    \hline
    PDF & cteq6l1 \todo{Citation} \\
    \hline
    \multicolumn{2}{|c|}{Cuts} \\
    \hline
    $\pt(\gamma)$ & $\SI{<20}{\giga \electronvolt}$ \\
    \hline
    $\Delta \mathrm{R}(\gamma,\mathrm{b})$ & $\num{<0.1}$ \\
    \hline
    \end{tabular}
     \label{tab_simu_2to5}
\end{table}

The resulting cross section is shown in Equation \ref{eq_simu_2to5_cross}. The cross section is of course larger than the one  of the $2 \to 3$ process (see Equation \ref{eq_simu_2to3_cross}), as more feynman diagrams are contributing to the final state. The difference between the cross sections obtained from MADGRAPH and WHIZARD of \SI{6}{\percent} is large compared to the respective uncertainties. It is probably due to the different factorisation scale described above.\\

\begin{equation}
\mathrm{WHIZARD: }\; \sigma_{\ttgamma} = \SI{908 \pm 2}{\femto \barn} \hspace{1.cm} \mathrm{MADGRAPH: }\; \sigma_{\ttgamma} = \SI{962.7 \pm 0.4}{\femto \barn}
\label{eq_simu_2to5_cross}
\end{equation}

Several observables are compared for MADGRAPH and WHIZARD, the result is shown in Figures \ref{fig_simu_comp_2to5_1} and \ref{fig_simu_comp_2to5_2}. They largely agree over a wide range. In the observable $\Delta \mathrm{R} (\gamma,\mathrm{b})$ there is a visible difference for low values (see Figure \ref{fig_simu_comp_2to5_drgb}). This might be due to different modelling for nearly colinear radiation. In any actual analysis the part of the phase space where these discrepancies exist will be cut by the selection or clustered into the jet.

\begin{figure}
  \subfigure[$\pt(\gamma)$]{
    \includegraphics[width = 0.67\textwidth] {Bilder/photonEt_Comp_2to5}
  }
  \\
    \subfigure[$\Delta \mathrm{R}(\mathrm{W},\gamma)$]{
    \includegraphics[width = 0.67\textwidth] {Bilder/dRphotonw_Comp_2to5}
  }
      \subfigure[$|\eta |(\gamma)$ \todo{Caption}]{
    \includegraphics[width = 0.67\textwidth] {Bilder/photonEta_Comp_2to5}
  }
  \caption{Comparison between the MADGRAPH and WHIZARD generators. All distributions are normalized to unity.}
  \label{fig_simu_comp_2to5_1}
\end{figure}

\begin{figure}
  \subfigure[$\Delta \mathrm{R} (\gamma,\mathrm{b})$]{
    \includegraphics[width = 0.67\textwidth] {Bilder/dRphotonb_Comp_2to5}
    \label{fig_simu_comp_2to5_drgb}
  }
  \\
    \subfigure[$\pt{\mathrm{W}^+}$]{
    \includegraphics[width = 0.67\textwidth] {Bilder/wpPt_Comp_2to5}
  }
      \subfigure[$\pt(\mathrm{b})$ \todo{Caption}]{
    \includegraphics[width = 0.67\textwidth] {Bilder/bquarkPt_Comp_2to5}
  }
  \caption{Comparison between the MADGRAPH and WHIZARD generators. All distributions are normalized to unity.}
  \label{fig_simu_comp_2to5_2}
\end{figure}

\subsection{The $2 \to 7$ Process}

As described in Section \ref{ch_simu_con} the final simulation uses a factorized $2 \to 7$ process. The details are shown in Equation \ref{eq_simu_fac}.\\
Again, the two generators MADGRAPH and WHIZARD are compared. Using WHIZARD the three processes can not be generated together, so they are generated separately and later added considering the scaling. Additionally, the "decay" syntax of WHIZARD requires looser cuts set for the decays and the final cuts being set as selection further complicating the generation. Consequently, MADGRAPH will be later used in Section \todo{Link}.\\
For simplicity reasons, only events where one of the top quarks decays into a muon (via the W boson) and the other decays into jets are used for the comparison. This is also the channel later used in this analysis in Section \todo{Link}. The full decay of the top is included in the generation, because the overall aim is to use the generated events for an official full simulation by the CMS collaboration. Therefore, other groups within CMS should be able to use this sample.\\
The parameters for the generation arrre shown in Table \ref{tab_simu_2to7}.

\begin{table}[ht]
\centering
    \caption{Technical parameters used for the generation of \ttgamma events (according to\\ Equation \ref{eq_simu_fac}. The scales are fixed in MADGRAPH. The parameters are the same for both generators.}
    \begin{tabular}{| l | l |}

    \hline
    Renormalization scale & $ \SI{174.3}{\giga \electronvolt} $ \\
    \hline
    Factorisation scale & $ \SI{174.3}{\giga \electronvolt} $ \\
    \hline
    PDF & cteq6l1 \todo{Citation} \\
    \hline
    \multicolumn{2}{|c|}{Cuts} \\
    \hline
    $\pt(\gamma)$ & $\SI{<13}{\giga \electronvolt}$ \\
    \hline
    $\pt(\mathrm{q})$ & $\SI{<15}{\giga \electronvolt}$ \\
    \hline
    $| \eta(\gamma)|$ & $\SI{<3}{}$ \\
    \hline
    $| \eta(\mathrm{l})|$ & $\SI{<3}{}$ \\
    \hline
    $| \eta(\mathrm{q})|$ & $\SI{<5}{}$ \\
    \hline
    $\Delta \mathrm{R}(\gamma,\mathrm{q/l})$ & $\num{<0.3}$ \\
    \hline
    \end{tabular}
     \label{tab_simu_2to7}
\end{table}

The Result for the generated cross section is shown in equation \ref{eq_simu_2to7_cross}. \\
The cross section generated by WHIZARD is given without an error. Due to the handling of decays in WHIZARD, the cross section calculated by the generator, does not apply the final cuts on the decay products, therefore the actual cross section of the decay has to be calculated using the number of remaining events. Additionally, the branching ratio of the simulated decay channels is not considered in the cross section results requiring further calculations by the user, especially in the channels wher a photon is required in the decay. \\
Of course all these calculation steps contribute to the uncertainty and their contributions would have to be estimated separately and then propagated to the final value. This is not done in this analysis. The cross sections are used here to compare the generators, therefore an exact calculation of these errors is deemed unnecessary. As mentioned before, a next to leading order cross section should be used in the analysis anyway. \\
The difference of $\SI{8}{\percent}$ between the two generators is therefore considered to be in the expected range.

\begin{equation}
\mathrm{WHIZARD: }\; \sigma_{\ttgamma} = \SI{1330}{\femto \barn} \hspace{1.cm} \mathrm{MADGRAPH: }\; \sigma_{\ttgamma} = \SI{1227.1 \pm 0.4}{\femto \barn}
\label{eq_simu_2to7_cross}
\end{equation}



\begin{figure}
  \subfigure[$\pt(\gamma)$]{
    \includegraphics[width = 0.67\textwidth] {Bilder/photonEt_2to7}
  }
  \\
    \subfigure[ $|\eta |(\gamma)$]{
    \includegraphics[width = 0.67\textwidth] {Bilder/photonEta_2to7}
  }
      \subfigure[ $\Delta \mathrm{R}(\mathrm{b},\gamma)$\todo{Caption}]{
    \includegraphics[width = 0.67\textwidth] {Bilder/dRphotonB_2to7}
    \label{fig_simu_comp_2to7_drgb}
  }
  \caption{Comparison between the MADGRAPH and WHIZARD generators. All distributions are normalized to unity.}
  \label{fig_simu_comp_2to7_1}
\end{figure}

\begin{figure}
  \subfigure[$\Delta \mathrm{R} (\gamma,\mathrm{q})$]{
    \includegraphics[width = 0.67\textwidth] {Bilder/dRphotonQ_2to7}
  }
  \\
    \subfigure[$\Delta \mathrm{R} (\gamma,\mathrm{l})$]{
    \includegraphics[width = 0.67\textwidth] {Bilder/dRphotonLepton_2to7}
  }
      \subfigure[$\pt(\mathrm{q})$ \todo{Caption}]{
    \includegraphics[width = 0.67\textwidth] {Bilder/quarkPt_2to7}
  }
  \caption{Comparison between the MADGRAPH and WHIZARD generators. All distributions are normalized to unity.}
  \label{fig_simu_comp_2to7_2}
\end{figure}
\begin{figure}
  \subfigure[$\mathrm{M}(\mathrm{q},\overline{\mathrm{q}})$]{
    \includegraphics[width = 0.67\textwidth] {Bilder/QQbarMass_2to7}
      }
  \\
    \subfigure[$\mathrm{M}(\gamma,\mathrm{l},\nu)$]{
    \includegraphics[width = 0.67\textwidth] {Bilder/photonLeptonNeutrinoMass_2to7}
  }
  \caption{Comparison between the MADGRAPH and WHIZARD generators. All distributions are normalized to unity.}
  \label{fig_simu_comp_2to7_3}
\end{figure}

The comparison for the two generators shown in Figures \ref{fig_simu_comp_2to7_1}, \ref{fig_simu_comp_2to7_2} and \ref{fig_simu_comp_2to7_3} displays compatibility over a long range of phase space. As before (see Figure \ref{fig_simu_comp_2to5_drgb}) the angle between photon and b quark $\Delta \mathrm{R}(\gamma,\mathrm{b})$ shows some discrepancies (see Figure \ref{fig_simu_comp_2to7_drgb}) for low values. This might be a genuine disagreement between the two generators, but because of the way jets and isolations are used at a proton proton collider experiment such as CMS these differences are irrelevant for any actual analysis.
Another reason might be a mistake in normalization of the three WHIZARD processes in respect to each other. \\

\section{Conclusion}

In Section \ref{ch_simu_comp} an agreement between the generators MADGRAPH and WHIZARD for multiple ways of simulating \ttgamma events and over a large amount of phase space was shown. The largest differences are found in the cross sections calculated by the two generators. For the factorized $2 \to 7$ approach they are probably due to technical difficulties when using WHIZARD for the simulation, but nevertheless the disagreement hints at the need for a dedicated NLO calculation. These NLO calculations typically calculate a factor to correct the cross section using distributions of kinematic observables \todo{Cite smth.}. The agreement in these observables between WHIZARD and MADGRAPH is therefore an important result.\\  
It is shown that WHIZARD and MADGRAPH can both be used to generate \ttgamma events. MADGRAPH is considerably faster and technically easier to use, whereas WHIZARD can easily be used to implement models including anomalous top couplings. \\
In Chapter \todo{Link} WHIZARD is used for a benchmark study estimating the sensitivity for anomalous top photon couplings. The $2 \to 7$ approach is used in MADGRAPH as the basis for a full CMS event simulation \todo{Link} including hadronization and detector response. A custom calculated cross section is provided. \todo{Some Citation} This new simulation should improve the systematic uncertainties for the $\sigma_{\ttgamma}$ measurement in CMS \todo{Link, maybe Value ?}. 
\chapter{Anomalous Couplings}

In this chapter, a benchmark study for anomalous couplings of the top quark is presented. A generator study is used to evaluate the sensitivity of a possible \ttgamma analysis for a non zero magnetic dipole moment of the top quark.
\section{The Top-Photon Vertex}

As previously mentioned \todo{Link} the \ttgamma Lagrangian can be described as shown in  Equation \ref{eq_ano_ttg} \cite{AguilarSaavedra:2008zc}.

\begin{equation}
  \mathcal{L}_{\ttgamma} = -e Q_t \overline{t} \gamma^{\mu} t A_{\mu} - e \overline{t} \frac{i \sigma^{\mu \nu}}{m_t} (d_V^{\gamma} + i d_a^{\gamma} \gamma_5) t A_{\mu}
  \label{eq_ano_ttg}
  \end{equation}

Here the first term is the contribution from the standard model, where the Lagrangian is proportional to the top-quark charge $Q_t$.\\
In the second term $d_a^{\gamma}$ and $d_V^{\gamma}$ are a possible electric- or  magnetic-dipole moment of the top quark, which would lead to loop corrections to the coupling strength of photons to top quarks. The electric-dipole moment is supposed to be heavily suppressed in the Standard Model \cite{Jersak:1981sp} and the magnetic-dipole moment is predicted to be zero. Consequently, any meaningful contribution should come from physics beyond the standard model. The magnetic-dipole moment was already discussed in another analysis \todo{Cite Markus}, therefore this analysis will look into a possible magnetic-dipole moment.\\
The  magnetic dipole moment can be connected to a hypothetical non-point like charge distribution, which would lead to a non zero radius of the top quark. Following \cite{Englert:2012by} \cite{Kopp:1994qv} the relation is shown in Equation \ref{eq_ano_rad}.

\begin{equation}
R_t = \frac{\sqrt{6}}{\Lambda_{\ast}} \hspace{1.cm} d_V^{\gamma} = \frac{\rho m_t^2}{\Lambda_{\ast}}
\label{eq_ano_rad}
\end{equation}

Here $\rho$ is a $\mathcal{O}(1)$ number, $\Lambda_{\ast}$ is the scale factor and $m_t$ is the top-quark mass.\\
A non zero magnetic-dipole moment could have an effect on \ttgamma events. In this chapter several observables are investigated in order to assess their sensitivity. The production cross section is also examined.

\section{Investigation using Simulated Events}

The generator WHIZARD is used to generate events for different top-quark magnetic dipole moments. The factorized $2 \to 7$ approach explained in Section \ref{ch_simu_con} and shown in Equation \ref{eq_ano_fac} is used.

\begin{equation}
\begin{split}
& \mathrm{pp}\to \ttgamma,  \mathrm{t} \to \mathrm{bxx},  \overline{\mathrm{t}} \to \overline{\mathrm{b}} \mathrm{xx} \\
& \mathrm{pp}\to \mathrm{t} \overline{\mathrm{t}},  \mathrm{t} \to \mathrm{bxx} \gamma,  \overline{\mathrm{t}} \to \overline{\mathrm{b}} \mathrm{xx} \\
& \mathrm{pp}\to \mathrm{t} \overline{\mathrm{t}},  \mathrm{t} \to \mathrm{bxx},  \overline{\mathrm{t}} \to \overline{\mathrm{b}} \mathrm{xx} \gamma \\
&  x\ = \ \mathrm{u}, \mathrm{d}, \mathrm{s}, \mathrm{c}, \overline{\mathrm{u}}, \overline{\mathrm{d}},  \overline{\mathrm{s}},  \overline{\mathrm{c}}, \mathrm{e}^{\pm}, \mu^{\pm}, \tau^{\pm}, \nu_{\mathrm{e},\mu,\tau}, \overline{\nu}_{\mathrm{e},\mu,\tau} 
\end{split}
\label{eq_ano_fac}
\end{equation}

These events also include photons which are not radiated from the top quark making them less sensitive for anomalous couplings, but it is difficult to separate these processes.\\
Only the output of the matrix element generator is taken into account. The detector response or any other uncertainties are not included here.
There should be a clear difference between the different scenarios, otherwise a real analysis using measured data is not reasonable.\\
The values chosen here for the multiple scenarios are shown in equation \ref{eq_ano_val}.

\begin{equation}
d_V^{\gamma} = 0.,\; 0.06,\; 0.12,\; 0.18,\; 0.30
\label{eq_ano_val}
\end{equation}

The highest value of $d_V^{\gamma} = 0.30$ corresponds to a "radius" of $R_{\mathrm{t}} = \SI{0.65d-3}{\femto \meter}$. One point of orientation for the choice where the limits on the magnetic dipole moment of the top quark \cite{Englert:2012by}, the other was the radius of the proton $R_{\mathrm{p}} \approx \SI{1}{\femto \meter}$. The proton radius should be significantly greater than the radius of the top quark. \\
As a first point for investigation the calculated cross sections for the $\mathrm{pp} \to \ttgamma $ process can be compared (see Table \ref{tab_ano_crosssec}). No errors are given here because of the technical complications when calculating these cross sections from the WHIZARD output (described in Section \ref{sec_simu_comp_2to7}). An error calculation is possible and would have to be included in an analysis, but for this benchmark study it is considered unnecessary.

\begin{table}[ht]
\centering
    \caption{Cross sections for the $\mathrm{pp} \to \ttgamma$ process as calculated by WHIZARD. Multiple scenarios for a possible anomalous magnetic dipole moment are compared. No errors are given.}
    \begin{tabular}{| l | l |}

    \hline
    $d_{V}^{\gamma}$ & $\sigma{\ttgamma} [\SI{}{\femto \barn}]$ \\
    \hline
    $0$ &  $1408$\\
    \hline
    $0.06$ & $1408 $ \\
    \hline
    $0.12$ & $1407 $ \\
    \hline
    $0.18$ & $1420 $ \\
    \hline
    $0.30$ &  $1443 $\\
    \hline
    \end{tabular}
     \label{tab_ano_crosssec}
\end{table}

The values for the cross section show little separation power, the first three being clearly within the uncertainty of the calculation. The difference in cross section between Standard Model ($d_V^{\gamma} = 0$)  and a scenario with $d_V^{\gamma} = 0.3$ is $\Delta \sigma_{\ttgamma} = \SI{35}{\femto \barn}$. As of now, such a difference cannot be measured with the CMS detector  ($\sigma^{CMS}_{\ttgamma} = \SI{2.4 \pm 0.6}{\pico \barn}$\cite{CMS-PAS-TOP-13-011}).\\
In the next step several kinematic observables are compared. They might offer more separation power than the comparison of the cross section.

\begin{figure}
\centering
  \subfigure[$\pt(\gamma)$]{
    \includegraphics[width = 0.67\textwidth] {Bilder/photonEt_Anomalous}
  }
  \\
    \subfigure[$\Delta \mathrm{R} (\gamma, \mathrm{q})$]{
    \includegraphics[width = 0.67\textwidth] {Bilder/dRphotonQ_Anomalous}
  }
      \subfigure[$\Delta \mathrm{R} (\gamma, \mathrm{t})$]{
    \includegraphics[width = 0.67\textwidth] {Bilder/dRphotonT_Anomalous}
  }
  \caption{Comparison between different scenarios for an anomalous electric dipole moment of the top quark. All distributions are normalized to unity.}
  \label{fig_ano_comp_1}
\end{figure}

The results are shown in Figure \ref{fig_ano_comp_1}. These comparisons show that the variables investigated here show no separation power.  The difference between the scenarios should be significant in order to at least have a chance of separating the two scenarios after the full detector simulation.\\
A few more complicated observables are investigated in Figure \ref{fig_ano_comp_2}. In Figure \ref{fig_ano_comp_2_1} the transversal momentum of the top quark is shown in two bins. A similar observable is used in another study on anomalous top couplings \todo{cite Markus}. In Figure \ref{fig_ano_comp_2_2} the angle between the two top quarks is drawn. This observable is inspired by spin-correlation measurements \cite{CMS-PAS-TOP-12-004} \cite{Höhle:1349025}. In a real analysis they would have to be reconstructed, which is difficult, or the observable would have to be changed to a angular correlation of the decay products.\\
Even with these observables the separation power does not increase significantly. For the phase space tested here, the data we currently have is insufficient for a measurement.   

\begin{figure}
\centering
  \subfigure[$\pt(\gamma)$]{
    \includegraphics[width = 0.67\textwidth] {Bilder/Rebin_Anomalous}
    \label{fig_ano_comp_2_1}
  }
  \\
    \subfigure[$\theta (\mathrm{t},\overline{\mathrm{t}})$]{
    \includegraphics[width = 0.67\textwidth] {Bilder/topangle_Anomalous}
    \label{fig_ano_comp_2_2}
  }
  \caption{Comparison between different scenarios for an anomalous electric dipole moment of the top quark. All distributions are normalized to unity.}
  \label{fig_ano_comp_2}
\end{figure}

To assess the impact of a very strong anomalous magnetic coupling, a few extreme scenarios were generated. The values for the electromagnetic coupling are shown in Equation \ref{eq_ano_val_ex}. They are unrealistic as such a high anomalous coupling would probably have been detected already.

\begin{equation}
d_V^{\gamma} = 0.,\; 0.30,\; 1.0,\;1.5,\; 5.0
\label{eq_ano_val_ex}
\end{equation}

Again, the cross sections are investigated (see Table \ref{tab_ano_crosssec_ex}). The different scenarios are now clearly distinct and it would be possible to set a limit on the coupling using the CMS result of $\sigma^{CMS}_{\ttgamma} = \SI{2.4 \pm 0.6}{\pico \barn}$\cite{CMS-PAS-TOP-13-011}. Taking into account the large corrections factors between leading order and next to leading order discussed in Chapter \todo{Link} setting limits is not feasible. The technical difficulties when calculating the cross section from the WHIZARD output also make these results less suited for a limit calculation. \\

\begin{table}[ht]
\centering
    \caption{Cross sections for the $\mathrm{pp} \to \ttgamma$ process as calculated by WHIZARD. Multiple scenarios for a possible very high anomalous magnetic dipole moment are compared. No errors are given. These are used to assess the general impact of a high anomalous coupling.}
    \begin{tabular}{| l | l |}

    \hline
    $d_{V}^{\gamma}$ & $\sigma{\ttgamma} [\SI{}{\femto \barn}]$ \\
    \hline
    $0$ &  $1408$\\
    \hline
    $0.3$ & $1443 $ \\
    \hline
    $1.0$ & $1723 $ \\
    \hline
    $1.5$ & $2161 $ \\
    \hline
    $5.0$ &  $12874 $\\
    \hline
    \end{tabular}
     \label{tab_ano_crosssec_ex}
\end{table}

Different kinematic observables are also studied (see Figure \ref{fig_ano_comp_ex}). The different scenarios are separable, especially the one with $d_{V}^{\gamma} = 5.0$ shows distinctive behavior. A limit calculation would be possible using either more different simulated scenarios and interpolation or  matrix element reweighing to achieve a continuous distribution. These methods would require a lot of effort and are not further pursued in this analysis.

\begin{figure}
\centering
  \subfigure[$\pt(\gamma)$]{
    \includegraphics[width = 0.67\textwidth] {Bilder/photonEt_Anomalous_extreme}
  }
  \\
    \subfigure[$\Delta \mathrm{R} (\gamma, \mathrm{q})$]{
    \includegraphics[width = 0.67\textwidth] {Bilder/dRphotonQ_Anomalous_extreme}
  }
      \subfigure[$\Delta \mathrm{R} (\gamma, \mathrm{t})$]{
    \includegraphics[width = 0.67\textwidth] {Bilder/dRphotonT_Anomalous_extreme}
  }
  \caption{Comparison between extreme scenarios for an anomalous electric dipole moment of the top quark. All distributions are normalized to unity.}
  \label{fig_ano_comp_ex}
\end{figure}

\section{Conclusion}

The benchmark study shows that the $\mathrm{pp} \to \ttgamma$ channel is currently not well suited for a high precision measurement of a possible magnetic moment of the top quark. While the investigation of a hypothetical very high coupling shows that a limit could be set using either the event kinematics or a more precisely calculated cross section, this would require plenty of time and effort. The result would likely be unsatisfactory, so a further investigation  currently does ot feel sensible. In all likelihood it will remain challenging even after the next LHC run has delivered its results. 
\chapter{A new Template Fit for the \ttgamma cross section measurement}

In this section the measurement of the \ttgamma cross section is discussed. It is heavily based on the analysis presented in \cite{CMS-PAS-TOP-13-011} and \cite{tholen:ma} and takes a closer look at certain aspects. The major change is the use of a different template in the template fit determining the final cross section. Nevertheless, the whole analysis is described here as it is necessary to understand the specific changes that were made.\\
As described in Section \todo{Link} a new factorized $2 \to 7$ has been developed for this measurement. However, the full simulation was not ready in time to be included here. As the full simulation has been performed by the CMS collaboration, the timescale is beyond the author's control \todo{First person ?}. Consequently, the $2 \to 5$ approach from \cite{CMS-PAS-TOP-13-011} described in Section \todo{Link} is used for the \ttgamma cross section measurement. 

\section{Analysis Strategy}
\subsection{Signal Process}
\label{sec_ttg_strat_sig}

\begin{figure}
\centering
    \includegraphics[width = 0.67\textwidth] {Bilder/feyn_ttgamma}
  \caption{Feynman graph of a \ttgamma event decaying into a muon and jets \cite{Hermanns:1292768}.}
  \label{fig_ttg_sig_fey}
\end{figure}


Figure \ref{fig_ttg_sig_fey} shows a Feynman diagram for the signal process investigated in this analysis. The \ttgamma decay process consists of a top-quark pair where one of the top quarks radiates a photon. In this analysis semi muonic events are inspected, so the signature can be described as follows:
\begin{itemize}
\item One high energetic muon
\item Four jets, two of which originate from a b quark
\item The neutrino resulting in missing transverse energy in the detector
\item A high energetic photon
\end{itemize}

\subsection{Background Processes}

The background processes can be divided into two parts. On the one hand there are processes which have a top-quark pair like signature, on the other hand there can be non signal type photon or photon like particle in a top-quark pair event.\\

\subsubsection{Top Pair like Background}

In order to have a top pair like signature events should mainly have one muon, missing transverse energy and four high energetic jets. The most important are shown in Figure \ref{fig_ttg_bkg} and discussed below.

\begin{itemize}
\item \textbf{W + Jets}: The signature consists of a muon and missing transverse energy from the W-boson decay, additional jets can be produced through QCD radiation (see Figure\ref{fig_ttg_bkg_wjets}). The more b jets are required the more the cross section is reduced. The jets typically have a lower energy than the ones originating from top-quark pair events.
\item \textbf{Drell Yan + Jets}: A photon or a Z boson decays into two muons and additional jets are produced (see Figure \ref{fig_ttg_bkg_dy}). One of the muons must be very low energetic or misidentified, otherwise the muon veto \todo{ Link ?} would reject the event. Missing transverse energy can be caused by particles not being detected or by mis-measured momenta. Tight requirements on the jets again reduce the cross section of this process.
\item \textbf{Single Top}: Single top-quark production can happen either in the s-channel, the t-channel or associated with a W boson (see Figure \ref{fig_ttg_bkg_singlet} for the s-channel). Even though the cross sections are low \todo{Cite ?} the very signal like structure allows for a significant contribution to the signal region.
\item \textbf{QCD Multijet}: This background is always a concern when dealing with a hadron collider because of its very large cross section. It is unlikely that a jet could be reconstructed as a muon making the process less relevant for the signal region (see Figure \ref{fig_ttg_bkg_qcd}).
\item \textbf{Other top-quark pair decays}: Other decay channels for top-quark pairs can also contribute in the signal region, due to their similar structure. Especially events where a tau lepton decays into a muon contribute. 
\end{itemize}

\begin{figure}[ht]
  \subfigure[W + Jets]{
    \includegraphics[width = 0.52\textwidth] {Bilder/feyn_wjets}
    \label{fig_ttg_bkg_wjets}
  }
    \subfigure[Drell-Yan + Jets]{
    \includegraphics[width = 0.52\textwidth] {Bilder/feyn_dy}
        \label{fig_ttg_bkg_dy}
 }
 \\
   \subfigure[Single Top (s-channel)]{
    \includegraphics[width = 0.52\textwidth] {Bilder/feyn_singlet}
        \label{fig_ttg_bkg_singlet}
  }
    \subfigure[QCD]{
    \includegraphics[width = 0.52\textwidth] {Bilder/feyn_qcd}
        \label{fig_ttg_bkg_qcd}
 }
  \caption{Example Feynman graphs for the \ttbar like background events. Additional jets are generated through quark or gluon radiation.\cite{steeger}}
  \label{fig_ttg_bkg}
\end{figure}

\subsubsection{Photon Signature Background}

\todo{Feynman Graphs + cite + links in text}

There are two different types of photon backgrounds. The first are different particles that are mis-reconstructed as photons. They might really be neutral hadrons, electrons, photons from a $\pi^0$ (these are of course actual photons, but they are not considered to be final state photons) or other particles. These mis-reconstructed particles are called fake photons throughout the analysis and they play a major role in the cross section measurement. \\
The second type of background are real photons that are radiated not from the top quark but either from initial state or different final state particles as shown in Figures \todo{Link}. Depending on the simulation strategy (see Section \todo{Link}) some of these events might actually belong to the signal region. For this measurement the $2 \to 5$ strategy is used for the signal simulation, so only top quarks radiated from the light quarks or the muon (see Figures \todo{Link}) are considered to contribute to the background region. Nevertheless, there are cuts in the selection (see Section \todo{Link}) to mitigate the influence of the modeling of initial and final state radiation.

\begin{figure}[ht]
  \subfigure[Quark Annihilation]{
    \includegraphics[width = 0.52\textwidth] {Bilder/feyn_ttgamma_isrqq}
    \label{fig_ttg_bkg_isrqq}
  }
    \subfigure[Gluon Fusion]{
    \includegraphics[width = 0.52\textwidth] {Bilder/feyn_ttgamma_isrgg}
        \label{fig_ttg_bkg_isrgg}
 }
  \caption{Background through additional photon radiation in \ttbar events. Here the photons are radiated from the initial state particles. Only quarks (a) in the initial state can radiate photons. \cite{Hermanns:1292768}}
  \label{fig_ttg_bkg_isr}
\end{figure}

\begin{figure}[ht]
  \subfigure[]{
    \includegraphics[width = 0.52\textwidth] {Bilder/fey_ttgamma_fsrb}
    \label{fig_ttg_bkg_fsrb}
  }
    \subfigure[]{
    \includegraphics[width = 0.52\textwidth] {Bilder/feyn_ttgamma_fsrw}
        \label{fig_ttg_bkg_fsrw}
 }
 \\
   \subfigure[]{
    \includegraphics[width = 0.52\textwidth] {Bilder/fey_ttgamma_fsrf1}
        \label{fig_ttg_bkg_fsrf1}
  }
    \subfigure[]{
    \includegraphics[width = 0.52\textwidth] {Bilder/fey_ttgamma_fsrf2}
        \label{fig_ttg_bkg_fsrf2}
 }
  \caption{Background through additional photon radiation in \ttbar events. Here the photon is radiated from one of the charged final state or intermediary particles. \cite{tholen:ma} \cite{Hermanns:1292768}}
  \label{fig_ttg_bkg_fsr}
\end{figure}

\subsection{Outline of the \ttgamma Cross Section Measurement}

Here a brief description of the analysis is given.\\

\begin{itemize}
\item Simulation of \ttgamma signal events including showering, hadronization and simulation of detector response (as described in Section \todo{Link}). Here the $2 \to 5$ simulation strategies is used. The simulated background events are provided by CMS.
\item A preselection of semi muonic top-quark pair events using the CMS reference selection \todo{Link}. Later the number of preselected events in data $N^{presel}$ and the purity of the selection calculated from simulated events $\pi_{\mathrm{t}\overline{\mathrm{t}}}$ are used.
\item Selection of events with a prompt photon \todo{Link}. The selection efficiency $\epsilon_{\gamma}$ is computed from simulation.
\item A template Fit to extract the number of real photons in the final candidates. Templates for real and fake photons are taken from data and fitted to a shower shape distribution (see Section \todo{Link}). The result of the fit is the number off selected \ttgamma events in data $N^{sig}_{\ttgamma}$.
\item The validity of the fit procedure is analysed with a closure test (see Section \ref{sec_ttg_clo}). Pseudo data is generated from simulated samples and then used for the template fit. The fit results should reproduce the input for the pseudo data generation.
\item The ratio between the \ttgamma and the $\mathrm{t}\overline{\mathrm{t}}$ cross section is calculated (see Equation ).

\begin{equation}
R \equiv \frac{R^{vis}}{\epsilon^{vis}_{\gamma}} = \frac{\sigma_{\ttgamma}}{1} \cdot \frac{1}{\sigma_{\mathrm{t}\overline{\mathrm{t}}}} = \frac{N^{sig}_{\ttgamma}}{\epsilon^{vis}_{\gamma} \epsilon_{\gamma}} \cdot \frac{1}{N^{presel} \pi_{\mathrm{t}\overline{\mathrm{t}}}}
\end{equation}

Other quantities, like the luminosity cancel out.
\item The \ttgamma cross section is then computed from the measured top-quark pair cross section (\todo{Link})

\begin{equation}
\sigma_{\ttgamma} = R \cdot \sigma_{\mathrm{t}\overline{\mathrm{t}}}
\end{equation}

\end{itemize}

\section{Preselection: Top-Quark Pair Events}

The preselection is based on CMS recommendations for a cut based top-quark pair event selection. A similar selection is implemented in \cite{CMS-PAS-TOP-12-027}. Here the selection requires exactly one high energetic muon, four high energetic jets and one of them should be b tagged. The details are found below.

\begin{itemize}
\item \textbf{High Level Trigger}: All events are required to have passed the IsoMu242p1 trigger path. It requires an isolated muon with a transverse momentum of $\pt > \SI{24}{\giga \electronvolt}$ and $| \eta | < 2.1$. The trigger is not prescaled.
\item \textbf{Tight Muon}: The events must have exactly one high energetic, isolated muon. The muons are required to have  $\pt > \SI{26}{\giga \electronvolt}$ to reach the plateau of the trigger efficiency and the pseudo rapidity $| \eta | <2.1 $. They are required to be global muons meaning they have to be reconstructed in the tracker as well as in the muon system. The relative isolation should be $I_{rel}(0.4) < 0.12$. Here, $I_{rel}$ is defined as the sum of the \pt of all neutral and charged particle candidates in a cone of $\Delta R < 0.4$ around the muon. The momentum connected to the muon is excluded and the isolation is corrected for pile up. The distance form the primary vertex is required to be $\Delta z < \SI{5}{\milli \meter}$ and $\Delta \rho < \SI{0.2}{\milli \meter}$. The combined efficiency of the muon trigger and selection is estimated to be $\SI{76}{\percent}-\SI{94}{\percent}$ contingent on the $\eta$ and \pt of the muon \cite{CMS-DP-2013-009} \cite{Chatrchyan:2012xi}.
\item \textbf{Primary Vertex}: All events should have at least one good primary vertex. The distance to the beam interaction region should be $\Delta z < \SI{24}{\centi \meter}$ and $\Delta \rho < \SI{2}{\centi \meter}$. If there are multiple vertices the one with the highest $\pt^2$ of all associated tracks is chosen as primary vertex. All leptons are required to originate from the primary vertex.
\item \textbf{Jets}: The events should have at least for jets with $| \eta | < 2.5$. The transverse momenta of the four leading jets should be larger then $\pt > 55,\; 45,\; 35, \; \SI{20}{\giga \electronvolt}$ respectively. 
\item \textbf{b-tag}: At least one of the four leading jets must be b tagged (see Section \todo{Link}). The combined secondary vertex algorithm with a value of CVSM$> 0.679$ is used. This leads to an efficiency of approximately \SI{70}{\percent} and a misidentification rate of approximately \SI{1.5}{\percent} \cite{CMS-PAS-TOP-12-027} \cite{Chatrchyan:2012jua}.
\item \textbf{Electron Veto}: Events with an electron candidate with $\mathrm{E}_{\mathrm{T}} > \SI{20}{\giga \electronvolt}$, $| \eta |< 2.5$ and $I_{rel}(0.3)< 0.15$ are rejected.
\item \textbf{Loose Muon Veto}: All Events with an additional global muon with $\pt > \SI{10}{\giga \electronvolt}$, $|\eta < 2.5|$ and $I_{rel}(0.4)<0.20$ are rejected. 
\end{itemize}

No minimum requirement on \ETm is applied, because of the jet energy scale leading to large uncertainties on \ETm .\\
Single top quark and W + jets events are the main background contributions. The number of preselected events is $N^{presel} =256665$ in data. The fraction of \ttbar events in the \ttgamma signal region is determined from simulation $\pi_{\ttbar} = N^{presel}_{\ttbar} / N^{presel} = \SI{84.3}{\percent}$. The simulated samples are normalized to data luminosity. About \SI{9}{\percent} of W + jets and \SI{5}{\percent} of single top quark events make up the rest of the preselected events.\\

\section{Photon Selection}

The photon candidates are reconstructed from significant energy deposits in the ECAL superclusters. Several isolation and kinematic variables are used for the cut based selection, details of which are found below. Influence of the underlying event and pile up on the isolation criteria are treated with the so called $\rho$ correction \cite{CMS-PAS-PFT-09-001}. The selection is based on the tight cut based photon identification used in \cite{CMS-PAS-HIG-13-006}. In order to pass the selection cuts the events are required to have at least one photon that fulfills the conditions.

\begin{itemize}
\item \textbf{Fiducialization}: The photon candidate is required to have a transverse energy $E_{\mathrm{T}} > \SI{25}{\giga \electronvolt}$ in order so suppress low energy fakes and photons not originating from the primary vertex. There is a further cut of $| \eta | < 1.4442$ to restrict the photons to the barrel region of the ECAL. The photon identification in the endcaps is not reliable due to the large material budget in front of it \cite{tholen:ma}.
\item \textbf{Electron veto}: The photon candidate is required to not have a track seed in the pixel detector. Due to the electron photon disambiguation employed in the particle reconstruction all photon candidates pass this cut.
\item \textbf{Tower-based H/E}: The ratio between the energy deposited in the HCAL towers behind the respective ECAL towers should be less than \SI{5}{\percent}
\item \textbf{Shower width $\sigma_{i \eta i \eta}$}: The definition of the shower shape variable $sigma_{i \eta i \eta}$ is shown in Equation \ref{eq_ttg_sel_sie}. The Condition is $\sigma_{i \eta i \eta} < 0.011$.
\begin{equation}
\sigma_{i \eta i \eta} = (\frac{\sum (\eta_i - \bar{eta})^2 \omega_i}{\sum \omega_i})^{1/2} ; \hspace{0.4cm} \bar{\eta} = \frac{\sum \eta_i \omega_i}{\sum \omega_i}; \hspace{0.4cm} \omega_i = max(0,4.7 + \log{\frac{E_i}{E_{5 \times 5}}})
\label{eq_ttg_sel_sie}
\end{equation}
\item \textbf{Neutral hadron isolation}: The sum of the \pt of neutral hadron candidates around the photon. The requirement is $I_{neu.had.} < \SI{3.5}{\giga \electronvolt} + 0.04 \times E_{\mathrm{T}}(\gamma)$  
\item \textbf{Photon isolation}: The sum of the \pt of other photon candidates around the photon. The requirement is $I_{photon} < \SI{1.3}{\giga \electronvolt} + 0.005 \times E_{\mathrm{T}}(\gamma)$. 
\item \textbf{FSR suppression}: In order to suppress final state radiation the photons are required to have a minimal distance from the final state particles. The condition is $\Delta R (\gamma,\mu /j) > 0.7$.
\end{itemize}

As seen in Table \ref{tab_ttg_sel_evt} there is a good agreement between data and simulation after the selection. The performance of the photon selection is satisfactory. The signal to background ratio is $S/B = 0.14$ and the \ttgamma selection efficiency is $\epsilon_\gamma = N^{sel}_{\ttgamma} / N^{presel}_{\ttgamma} = \SI{62.9}{\percent}$. The total number of selected data events is $N^{sel} = 7836$. This number should be sufficient to perform a template fit.

\begin{landscape}
\begin{table}
\caption{Number of events per sample after the selection steps of the prompt photon selection. The uncertainties given here are statistical errors.}
\begin{tabular}{l | r r r r r r r r r r r }

                  &           presel. &           $E_{T}$ &          $|\eta|$ &      $e^\pm$ veto &             H / E & $\sigma_{i\eta i\eta}$ &         hadr. iso &        neutr. iso &          pho. iso & $\Delta R(\gamma, \mu)$ & $\Delta R(\gamma, j)$ \\

\hline

\hline

         whiz2to5 &$            1872.8 $&$            1737.3 $&$            1508.7 $&$            1508.7 $&$            1444.5 $&$            1313.2 $&$            1020.6 $&$            1009.7 $&$             950.4 $&$             884.6 $&$             852.7 $ \\

                  &$ \pm              8.1 $&$ \pm              7.8 $&$ \pm              7.3 $&$ \pm              7.3 $&$ \pm              7.1 $&$ \pm              6.8 $&$ \pm              6.0 $&$ \pm              6.0 $&$ \pm              5.8 $&$ \pm              5.6 $&$ \pm              5.5 $ \\

             T\_t &$            2275.4 $&$            1258.0 $&$             773.2 $&$             773.2 $&$             621.3 $&$             132.8 $&$              33.5 $&$              24.2 $&$              13.8 $&$              13.8 $&$              13.8 $ \\

                  &$ \pm            159.1 $&$ \pm            118.3 $&$ \pm             92.8 $&$ \pm             92.8 $&$ \pm             83.2 $&$ \pm             38.4 $&$ \pm             19.3 $&$ \pm             16.4 $&$ \pm             12.4 $&$ \pm             12.4 $&$ \pm             12.4 $ \\

            T\_tW &$            4088.2 $&$            2148.2 $&$            1735.5 $&$            1735.5 $&$            1305.8 $&$             511.4 $&$              66.6 $&$              60.6 $&$              44.6 $&$              26.8 $&$              26.8 $ \\

                  &$ \pm             42.4 $&$ \pm             30.7 $&$ \pm             27.6 $&$ \pm             27.6 $&$ \pm             24.0 $&$ \pm             15.0 $&$ \pm              5.4 $&$ \pm              5.2 $&$ \pm              4.4 $&$ \pm              3.4 $&$ \pm              3.4 $ \\

          Tbar\_t &$            1267.7 $&$             667.1 $&$             459.8 $&$             459.8 $&$             343.2 $&$             130.2 $&$              10.9 $&$              10.5 $&$               7.5 $&$               4.8 $&$               4.8 $ \\

                  &$ \pm             19.9 $&$ \pm             14.4 $&$ \pm             12.0 $&$ \pm             12.0 $&$ \pm             10.4 $&$ \pm              6.4 $&$ \pm              1.8 $&$ \pm              1.8 $&$ \pm              1.5 $&$ \pm              1.2 $&$ \pm              1.2 $ \\

         Tbar\_tW &$            4164.1 $&$            2148.8 $&$            1746.5 $&$            1746.5 $&$            1311.6 $&$             525.9 $&$              52.8 $&$              48.1 $&$              34.9 $&$              24.6 $&$              24.6 $ \\

                  &$ \pm             43.0 $&$ \pm             30.9 $&$ \pm             27.8 $&$ \pm             27.8 $&$ \pm             24.1 $&$ \pm             15.3 $&$ \pm              4.8 $&$ \pm              4.6 $&$ \pm              3.9 $&$ \pm              3.3 $&$ \pm              3.3 $ \\

           DYJets &$            2961.0 $&$            1458.4 $&$            1058.4 $&$            1058.4 $&$             805.5 $&$             374.6 $&$             122.3 $&$             114.0 $&$             100.2 $&$              90.1 $&$              90.1 $ \\

                  &$ \pm             82.1 $&$ \pm             57.6 $&$ \pm             49.1 $&$ \pm             49.1 $&$ \pm             42.8 $&$ \pm             29.2 $&$ \pm             16.7 $&$ \pm             16.1 $&$ \pm             15.1 $&$ \pm             14.3 $&$ \pm             14.3 $ \\

            WJets &$           22662.1 $&$           10925.7 $&$            8300.0 $&$            8300.0 $&$            5809.2 $&$            2165.4 $&$             262.1 $&$             202.3 $&$             161.8 $&$             116.7 $&$             116.7 $ \\

                  &$ \pm            538.8 $&$ \pm            374.1 $&$ \pm            326.1 $&$ \pm            326.1 $&$ \pm            272.8 $&$ \pm            166.6 $&$ \pm             57.9 $&$ \pm             50.9 $&$ \pm             45.5 $&$ \pm             38.7 $&$ \pm             38.7 $ \\

           TTMadG &$          205099.4 $&$          106788.6 $&$           86539.9 $&$           86539.9 $&$           68074.9 $&$           27888.3 $&$            2639.0 $&$            2398.9 $&$            1633.6 $&$            1261.9 $&$            1248.2 $ \\

                  &$ \pm            386.7 $&$ \pm            279.0 $&$ \pm            251.2 $&$ \pm            251.2 $&$ \pm            222.8 $&$ \pm            142.6 $&$ \pm             43.9 $&$ \pm             41.8 $&$ \pm             34.5 $&$ \pm             30.3 $&$ \pm             30.2 $ \\

\hline

           MC Sum &$          244390.8 $&$          127132.0 $&$          102121.9 $&$          102121.9 $&$           79716.0 $&$           33042.0 $&$            4207.7 $&$            3868.2 $&$            2946.7 $&$            2423.2 $&$            2377.6 $ \\

                  &$ \pm            690.0 $&$ \pm            487.2 $&$ \pm            426.8 $&$ \pm            426.8 $&$ \pm            366.2 $&$ \pm            225.7 $&$ \pm             77.6 $&$ \pm             70.4 $&$ \pm             61.0 $&$ \pm             53.2 $&$ \pm             53.1 $ \\

\hline

             RunA &$             11221 $&$              5848 $&$              4674 $&$              4674 $&$              3706 $&$              1523 $&$               217 $&$               200 $&$               146 $&$               125 $&$               122 $ \\

             RunB &$             57067 $&$             29608 $&$             23538 $&$             23538 $&$             18598 $&$              7797 $&$              1086 $&$               993 $&$               725 $&$               583 $&$               575 $ \\

             RunC &$             92718 $&$             48350 $&$             38650 $&$             38650 $&$             30711 $&$             12785 $&$              1755 $&$              1630 $&$              1150 $&$               952 $&$               946 $ \\

             RunD &$             95659 $&$             49461 $&$             39548 $&$             39548 $&$             31412 $&$             12965 $&$              1812 $&$              1673 $&$              1203 $&$               993 $&$               983 $ \\

\hline

         Data Sum &$            256665 $&$            133267 $&$            106410 $&$            106410 $&$             84427 $&$             35070 $&$              4870 $&$              4496 $&$              3224 $&$              2653 $&$              2626 $ \\

\hline

\end{tabular}
\label{tab_ttg_sel_evt}
\end{table}
\end{landscape}

\section{Analyzing the Real Photon Content}

As mentioned in Section \todo{Link} the cut based identification does not sufficiently reduce the number of non photon particles identified as a photon. Therefore a template fit is applied in order to determine the number of events with actual photons. The focus of the analysis lies in this part of the measurement introducing a new template to have a completely data-driven template fit.\\
For the fit, templates for real and fake photons are taken from data and then fitted to the distribution for charged hadron isolation. The template for fake photons is taken from a data sideband, whereas the template for real photons is obtained by using the so called random cone isolation.\\
The purity of the \ttgamma events is calculated from simulation.\\

\subsection{Photon Matching}
d
Because the two classes of real and fake photons are mentioned often in this chapter, they are shortly explained here. For simulated events it is possible to assign reconstructed photons to generated photons or other generated particles to make sure whether a reconstructed photon is a real photon. Photons that are matched to a generated photon that is not part of a jet, which means it does not have a hadronic origin are classified as real photons. Photons that are not matched according to the above criteria are classified as fake photons.\\
The details of the matching procedure are as follows: The generated and reconstructed photons have to be within a cone of $\Delta R < 0.2$ to each other. The energy difference must be $\Delta E / E < \SI{100}{\percent}$. Should more than one photon be matched then the one with the smallest $\Delta R$ is taken.  \todo{Plot on performance ?}

\subsection{The Template Fit}

In this section the template fit is discussed. First the observable that is used for the discrimination between fake and real photons is described. Then the way the templates for real and fake photons are obtained is explained. Then the result of the template fit is presented and discussed. At last the performance of the template fit is tested with a closure test. \\

\subsubsection{The Discriminating Observable: Charged Hadron Isolation}

The charged hadron isolation is similar to the other particle based isolations described in Section \todo{Link}. It is the sum of the \pt of all charged particle candidates in a cone of $\Delta R = 0.4$ around the photon. It should therefore be a good discriminator against jets misidentified as photons, which should make up a large number of the faked photons, because the investigated \ttgamma events are required to have a large amount of jets.\\  
Figure \ref{fig_ttg_fit_chhadnm1} shows the distribution of the charged hadron isolation after the selection cuts. The agreement between data and simulation is acceptable. There is a significant disagreement between data and simulation in the first bin that should be signal dominated. This is one of the reasons why a fully data driven fit should yield a more reliable result.\\
For the template fit only the photon with the highest transverse energy is considered for each event. Additionally, particles pointing to the supercluster of the photon are removed from the calculation of the isolation. This so called supercluster footprint removal is used to minimize the correlation between the charged hadron isolation and the shower shape observable $\sigma_{i \eta i \eta}$ \cite{CMS-PAS-HIG-13-006}.\\

\begin{figure}
\centering
    \includegraphics[width = 0.67\textwidth] {Bilder/Nm1_ChHadIso}
  \caption{Logarithmic plot of the charged hadron isolation after the selection. The simulated samples are normalized to data luminosity.}
  \label{fig_ttg_fit_chhadnm1}
\end{figure}

\subsubsection{The Background Template}
\label{sec_ttg_fit_bkg}

The background template is taken from a data sideband as described in \cite{CMS-PAS-TOP-13-011}. Three selection requirements are utilized, the method is called SBID in this analysis. Photon candidates are selected for this sideband if they fail at least one of the three following conditions (see Section \todo{Link} for a more in depth explanation of the selection): 
\begin{itemize}
\item $\sigma_{i \eta i \eta} < 0.011$
\item $I_{neu.had.} < \SI{3.5}{\giga \electronvolt} + 0.04 \times E_{\mathrm{T}}(\gamma)$ 
\item $I_{photon} < \SI{1.3}{\giga \electronvolt} + 0.005 \times E_{\mathrm{T}}(\gamma)$
\end{itemize}

An alternative sideband distribution is later used for comparison. It only has the requirement to have $0.012 < \sigma_{i \eta i \eta} < 0.018$. The number of photons in this sideband region is comparatively small.\\
The two sideband distributions are compared with the photon fake distribution from the simulated \ttbar sample (the sideband distributions are also taken from the same sample) in Figure \ref{fig_ttg_fit_faketemplcomp}. The distributions are rebinned for the fitting in order to mitigate the impact of low statistics for higher values. There are differences between the fake and the sideband distributions, which could be due to the modeling in the simulation. In the previous version of this analysis \cite{CMS-PAS-TOP-13-011} the sideband distributions from simulation where weighted bin by bin to look like the fake distributions. These weights were then propagated to the data distribution and their impact treated as a systematic uncertainty. This reweighing is not applied here. The impact on the template fit will be discussed later \todo{Link}.

\begin{figure}
\centering
    \includegraphics[width = 0.67\textwidth] {Bilder/FakeTemplComp}
  \caption{Comparison of the fake and sideband distributions for charged hadron distribution from the simulated \ttbar sample. The distributions are normalized to unity.}
  \label{fig_ttg_fit_faketemplcomp}
\end{figure}

\subsubsection{The Signal Template}
\label{sec_ttg_fit_sig}

In order to get a data driven signal template a technique called random cone isolation is used \cite{Chatrchyan:2011qt}. The purpose is to obtain a signal like distribution that should be independent from the selection. Starting from a reconstructed photon the following steps are used (see also Figure \ref{fig_ttg_fit_raco}):\\
\begin{itemize}
\item Starting from the direction of a photon a new axis (the "random-cone axis") is generated by rotating around the photon axis in $\phi$ direction by a random angle between $0.8$ and $2 \pi - 0.8$. This ensures that an isolation cone around the new axis will not overlap with an isolation cone around the axis of the original photon.
\item The isolation of the new axis is checked. No particle flow jet with $\pt > \SI{20}{\giga \electronvolt}$ should have a $\Delta R < 0.8$. No photon with $\pt > \SI{10}{\giga \electronvolt}$ should have $\Delta R < 0.8$ and no muon should have a $\Delta R < 0.4$. If any of these conditions is not fulfilled than a new random axis is determined according to the criteria mentioned above. 
\item The charged hadron isolation of the new axis is calculated. For the calculation an area corresponding to the removed footprint of the original photon is removed from the calculation of the isolation of the new axis. This is done by rotating the supercluster of the original photon by the same angle the axis was rotated. Thereby the removed area is exactly he same as the original removed supercluster.
\end{itemize}

\begin{figure}[ht]
  \subfigure[]{
    \includegraphics[width = 0.4\textwidth] {Bilder/IsolationCone}
  }
    \subfigure[]{
    \includegraphics[width = 0.55\textwidth] {Bilder/RandomConeIso}
  }
  \caption{Sketch to explain the random cone technique. If the red circle represents the footprint of the photon in picture a) and is then completely removed in picture b), the green circles representing the isolation should be the same for both picture a) and b). Consequently, the random cone isolation predicts the original isolation. \cite{RandCone_Talk} \cite{RandCone_AN}}
  \label{fig_ttg_fit_raco}
\end{figure}

In the end the value of new isolation calculated with the random cone technique should not depend on the specific photon it was calculated from, but rather on the overall event shape. Effects like pile up, the underlying event or ECAL noise should impact the new isolation.\\
The charged hadron isolation calculated with the random cone technique (this variable is called random cone isolation during the analysis) in data is compared to the charged hadron isolation of the matched real photons in the simulated \ttgamma sample and with the random cone isolation in the \ttgamma sample in Figure \ref{fig_ttg_fit_raco}.\\
The random cone isolation shows little change between simulation and data, but there is a significant deviation in the first bin. \\
The templates of random cone isolation and charged hadron isolation of real photons are not compatible. It is hard to tell whether this is due to mismodelling in the simulation or if the random cone technique does not work in this case. As mentioned in Section \ref{sec_ttg_fit_bkg} there is no reweighing applied here for the template fit, so the impact of the discrepancies will be discussed in Section \todo{Link}.


\begin{figure}[ht]
  \subfigure[Data: $I_{rand.co.}$, MC: $I_{ch.had.}$]{
    \includegraphics[width = 0.52\textwidth] {Bilder/RandCone_Comp_datamc_lin}
  }
    \subfigure[Data: $I_{rand.co.}$, MC: $I_{ch.had.}$]{
    \includegraphics[width = 0.52\textwidth] {Bilder/RandCone_Comp_DataMc_log}
 }
   \subfigure[Data: $I_{rand.co.}$, MC: $I_{rand.co.}$]{
    \includegraphics[width = 0.52\textwidth] {Bilder/RandCone_Comp_datamc2_lin}
  }
    \subfigure[Data: $I_{rand.co.}$, MC: $I_{rand.co.}$]{
    \includegraphics[width = 0.52\textwidth] {Bilder/RandCone_Comp_DataMc2_log}
 }
  \caption{Comparison of the charged hadron isolation calculated with the random cone technique in data with the charged hadron isolation of real photons in the \ttgamma simulation in a) and b). In c) and d) the isolations computed via the random cone are compared between the \ttgamma simulation and data.}
  \label{fig_ttg_fit_raco}
\end{figure}

\subsection{The Result of the Template Fit}
\label{sec_ttg_fit_res}

The template fit itself is performed with the theta framework \cite{theta} using a binned maximum likelihood fit. For the calculation of the uncertainty the Barlow-Beeston light method is used \cite{Barlow:1993dm} \cite{2011arXiv1103.0354C} expecting the uncertainty for each bin to be Gaussian. The result of the template fit together with the two templates for fitting is shown in Figure \ref{fig_ttg_fit_res}.\\
The result shows agreement between the fitted templates and data considering the uncertainties. This is  reflected in the $\chi^2 / NDF = 1.2$. The result (shown in Equation \ref{eq_ttg_fit_res}) is compatible with the previous result \cite{CMS-PAS-TOP-13-011} while having a smaller statistical error. \\

\begin{equation}
N_{real} = 1894 \pm 76 \hspace{1cm} N_{real,prev} = 1761 \pm 120
\label{eq_ttg_fit_res}
\end{equation}

For the cross section calculation the number of real photons has to be corrected for background events using the purity of the selection.
The purity and the corrected result are shown in Equation \ref{eq_ttg_fit_rescor}.\\

\begin{equation}
\pi_{\ttgamma} = N^{sel}_{sig} / N^{sel} = \SI{66.7}{\percent} \hspace{1cm} N^{sig}_{\ttgamma} = 1263 \pm 51
\label{eq_ttg_fit_rescor}
\end{equation}

The remaining question is the validity of the fit procedure. The discrepancies between the templates used in the fit and the templates using the matched real and fake photons from simulation (as described in Sections \ref{sec_ttg_fit_bkg} and \ref{sec_ttg_fit_sig}) should be investigated. To test the validity closure tests are used. The results are described in the following (see Section \todo{Link}).

\begin{figure}[ht]
\centering
  \subfigure[]{
    \includegraphics[width = 0.67\textwidth] {Bilder/TemplateFit_Templates}
  }
  \\
    \subfigure[]{
    \includegraphics[width = 0.67\textwidth] {Bilder/TemplateFit_Result}
  }
  \caption{a) The templates used for the template fit normalized to integral. The two templates are described in detail in Sections \ref{sec_ttg_fit_bkg} and \ref{sec_ttg_fit_sig}.\\ b) The result of the template fit. The two templates are scaled to the fit parameters $N_{real}$ and $N_{fake}$}
  \label{fig_ttg_fit_res}
\end{figure}

\subsection{Assessing the Template Fit: Closure Tests}
\label{sec_ttg_clo}

The purpose of the closure test is to validate a certain fit method by using pseudo experiments. Pseudo data is produced with a set of known parameters and the fit should be able to reproduce these parameters. In detail, specifics used in this analysis are as follows:

\begin{itemize}
\item The pseudo data is obtained from two different sets of samples. One set consists of the simulated \ttgamma sample and the \ttbar background sample. The other consists of \ttgamma + jets and a \ttbar + jets sample that also include out of time pile up (explained in Section \todo{Link}).
\item For both sets the pseudo data is generated by using two templates of matched real and fake photons. The template for the real photons is taken from the \ttgamma sample and the fake template is taken from the \ttbar sample. These two templates are then mixed with factors that are varied around the result from the template fit on data (see Section \ref{sec_ttg_fit_res}). The result for each set of two factors is then taken as pseudo data. The errors on the pseudo data are statistical in respect to the final value, using Poisson errors for each bin.
\item The pseudo data from the last step is then fitted with the random cone template taken from the \ttgamma sample as a template for real photons and the sideband template from the \ttbar sample as a template for fake photons. The uncertainties of the templates used for fitting are only statistical errors.
\item The results should then reproduce the factors used for the mixing of the pseudo data. 
\end{itemize}

\subsubsection{The First Closure Test}
\label{sec_ttg_clo_norm}

This closure test is performed using the \ttgamma and \ttbar samples without out of time pile up. An example result is shown in Figure \ref{fig_ttg_clo_ex1}. There are significant differences between the pseudo data and the result of the fit. Another problem is the huge error on the result for the number of fake photons. A reason might be that there are generally more events in the templates taken from simulation compared to the templates taken from data. Therefore the statistical error on the templates used for fitting in this closure test is lower than the statistical error in the templates used in the templates from data. Still, the result approximately reproduces the factors used for fitting.\\
The comprehensive results for the whole closure test are shown in Figure \ref{fig_ttg_clo_seqnorm}.\\
The huge errors on the number of fake photons is again visible, but it seems to be less pronounced in the second sequence where the number of fake photons is varied (see Figures \ref{fig_ttg_clo_seqnorm21} and \ref{fig_ttg_clo_seqnorm22}). This might be due to a technical problem due to the normalization of the templates used for the fit. This needs to be further investigated, but this is not possible here due to time constraints.\\
The results of the whole closure test mostly reproduce the input parameters at least within two times the statistical uncertainty. There is however a systematic structure in the results of the closure test. The reason might be an insufficient separation between signal and background templates. This will be further investigated in Section \todo{Link}.  

\begin{figure}[ht]
\centering

    \includegraphics[width = 0.67\textwidth] {Bilder/ClosureTest_norm_ex}
    
  \caption{An example for the result of the closure Test without out of time pile up. The "mix" entry in the legend shows the factors used for the mixing of the pseudo data. The first number is the factor for fake photons, the second for real photons. }
  \label{fig_ttg_clo_ex1}
\end{figure}

\begin{figure}[ht]
  \subfigure[Fake photons]{
    \includegraphics[width = 0.52\textwidth] {Bilder/ClosureTest_norm_seq11}
  }
    \subfigure[Real photons]{
    \includegraphics[width = 0.52\textwidth] {Bilder/ClosureTest_norm_seq12}
 }
   \subfigure[Fake photons]{
    \includegraphics[width = 0.52\textwidth] {Bilder/ClosureTest_norm_seq21}
    \label{fig_ttg_clo_seqnorm21}
  }
    \subfigure[Real photons]{
    \includegraphics[width = 0.52\textwidth] {Bilder/ClosureTest_norm_seq22}
    \label{fig_ttg_clo_seqnorm22}
 }
  \caption{The results of the closure test for real and fake photons without out of time pile up. In the sequence shown in a) and b) the number of real photons is varied, in sequence c) and d) the number of fake photons is varied.}
  \label{fig_ttg_clo_seqnorm}
\end{figure}

\subsubsection{The Second Closure Test}
\label{sec_ttg_clo_oot}

This closure test is performed using the \ttgamma + jets and the \ttbar +jets samples with out of time pile up. An example result is shown in Figure \ref{fig_ttg_clo_ex2}. In contrast to the first closure test (see Section \ref{sec_ttg_clo_norm}) the fitted templates and the pseudo data agree with each other within their uncertainties. The huge error on the number of fake photons persist, which is a hint that it may be due to the normalization or the statistical errors on the fitted templates. Since the huge errors occur for both template fits, which are independent of each other, but technically done in the same way, the reason seems to lie in the technical implementation. \\
The factors used for the mixing of the pseudo data are not reproduced by the fit results. While the number of fake photons is reproduced within the uncertainty, this is not meaningful, because of the huge uncertainties. The number of real photons seems to be significantly under fitted. \\
This underfitting of the number of real photons becomes more apparent when looking at the whole closure test shown in Figure \ref{fig_ttg_clo_seqoot}. The number of real photons is consistently underfitted for both sequences. \\
The systematic structure visible in the closure test without out of time pile up(see Figure \ref{fig_ttg_clo_seqnorm}) is less pronounced 
in the closure test with out of time pile up. It should be due to a better separation between real and fake photons. Another reason could be the better agreement between the fitted templates and the pseudo data leading to a more stable fit result. This is further discussed in Section \todo{Link}.\\

\begin{figure}[ht]
\centering

    \includegraphics[width = 0.67\textwidth] {Bilder/ClosureTest_oot_ex}
    
  \caption{An example for the result of the closure Test with out of time pile up. The "mix" entry in the legend shows the factors used for the mixing of the pseudo data. The first number is the factor for fake photons, the second for real photons. }
  \label{fig_ttg_clo_ex2}
\end{figure}

\begin{figure}[ht]
  \subfigure[Fake photons]{
    \includegraphics[width = 0.52\textwidth] {Bilder/ClosureTest_oot_seq11}
  }
    \subfigure[Real photons]{
    \includegraphics[width = 0.52\textwidth] {Bilder/ClosureTest_oot_seq12}
 }
   \subfigure[Fake photons]{
    \includegraphics[width = 0.52\textwidth] {Bilder/ClosureTest_oot_seq21}
    \label{fig_ttg_clo_seqoot21}
  }
    \subfigure[Real photons]{
    \includegraphics[width = 0.52\textwidth] {Bilder/ClosureTest_oot_seq22}
    \label{fig_ttg_clo_seqoot22}
 }
  \caption{The results of the closure test for real and fake photons without out of time pile up. In the sequence shown in a) and b) the number of real photons is varied, in sequence c) and d) the number of fake photons is varied.}
  \label{fig_ttg_clo_seqoot}
\end{figure}

\subsubsection{Further Tests}
\todo{New Section Name}

In order to further investigate the validity of the fit more tests are perormed.\\
As a sanity check the closure test is repeated, but this time the templates used for the generation of the pseudo data are also used for the template fit. The result should exactly reproduce the input, as the templates are fitted to themselves.\\
The results are shown in Figure \ref{fig_ttg_clo_toy}. The fit results exactly reproduce the input, while the errors on the number of fake photons is still huge. This shows that the fit technically works on a basic level.\\

\begin{figure}[ht]
  \subfigure[without OOT pile up]{
    \includegraphics[width = 0.52\textwidth] {Bilder/ClosureTest_toy_ex2}
  }
    \subfigure[with OOT pile up]{
    \includegraphics[width = 0.52\textwidth] {Bilder/ClosureTest_toy_ex1}
 }
   \subfigure[without OOT pile up]{
    \includegraphics[width = 0.52\textwidth] {Bilder/ClosureTest_toy_seq2}
    \label{fig_ttg_clo_toy_seq2}
  }
    \subfigure[with OOT pile up]{
    \includegraphics[width = 0.52\textwidth] {Bilder/ClosureTest_toy_seq1}
    \label{fig_ttg_clo_toy_seq1}
 }
  \caption{a) and b) show examples for the fit with and without out of time pile up. c) and d) show the results for the overall closure test with and without out of time pile up.}
  \label{fig_ttg_clo_toy}
\end{figure}

As a further test the template for fake photons used for the fit is changed to the alternative template described in Section \todo{Link}. Their is little change in the results of the template fit, as can be seen on Figure \ref{fig_ttg_clo_chte}. The systematic effect for the samples without out of time pile up is still significant (see Figure \ref{fig_ttg_clo_chte1}). This shows that the separation between the templates for real and fake photons is likely not the cause for this effect, as the alternative fake template should provide a different degree of separation.\\

\begin{figure}[ht]
   \subfigure[without OOT pile up]{
    \includegraphics[width = 0.52\textwidth] {Bilder/ClosureTest_norm_ChTe}
    \label{fig_ttg_clo_chte1}
  }
    \subfigure[with OOT pile up]{
    \includegraphics[width = 0.52\textwidth] {Bilder/ClosureTest_norm_ChTe2}
 }
  \caption{Results of the closure test with an alternative template used to fit the amount of fake photons.}
  \label{fig_ttg_clo_chte}
\end{figure}

For the next test the templates used for the fit and the pseudo data generation were interchanged: The \ttgamma + jets sample with out of time pile up was used with the \ttbar sample (see Figures \ref{fig_ttg_clo_test_11} and \ref{fig_ttg_clo_test_12}) and the \ttgamma sample was used with the \ttbar + jets sample with out of time pile up (see Figures \ref{fig_ttg_clo_test_21} and \ref{fig_ttg_clo_test_22}). The results show that the combination of \ttgamma + jets and \ttbar samples lead to overfitting the number of real photons instead of underfitting them. The combination of the \ttgamma with the \ttbar + jets sample leads to a more significant underfitting. This shows that the differences from expectation shown in Section \ref{sec_ttg_clo_oot} cannot be pinned down to a single sample where there something might be wrong, but rather that it is the combination of the samples that lead to the discrepancies.

\begin{figure}[ht]
  \subfigure[\ttgamma + jets with \ttbar]{
    \includegraphics[width = 0.52\textwidth] {Bilder/Closure_Test_oot_testmc}
    \label{fig_ttg_clo_test_11}
  }
    \subfigure[\ttgamma + jets with \ttbar]{
    \includegraphics[width = 0.52\textwidth] {Bilder/Closure_Test_oot_testmc2}
    \label{fig_ttg_clo_test_12}
 }
   \subfigure[\ttgamma with \ttbar +jets]{
    \includegraphics[width = 0.52\textwidth] {Bilder/Closure_Test_oot_test2mc}
    \label{fig_ttg_clo_test_21}
  }
    \subfigure[\ttgamma with \ttbar +jets]{
    \includegraphics[width = 0.52\textwidth] {Bilder/Closure_Test_oot_test2mc2}
    \label{fig_ttg_clo_test_22}
 }
  \caption{The results of the closure tests with interchanged templates. The \ttgamma + jets and \ttbar +jets samples include out of time pile up.}
  \label{fig_ttg_clo_test}
\end{figure}

\subsubsection{Conclusion}

The tests that are performed to understand the unsatisfactory result of the closure test (with out of time pile up, see Section\ref{sec_ttg_clo_oot}) provide no satisfying reason. One reason could be the simulation mis-modeling the charged hadron isolation. This can be motivated by the comparison between simulation and data in Figure \ref{fig_ttg_fit_chhadnm1} where the photons in simulation seem to be less isolated than the ones in data, even without additional jets or out of time pile up. \\
The other assumption would be that the random cone isolation technique does not work in this case, as evidenced by the disagreement between the template obtained with random cone isolation and the template of matched real photons obtained from simulation (see Figure \ref{fig_ttg_fit_raco}). \\ 
Further tests are neccessary make a decision on the validity if the technique presented here. Because of time constraints, they could not be investigated. In order to illustrate what the end result looks like, the analysis is continued by investigating the systematic uncertainties and giving a final result.

\section{Systematic Uncertainties}

The systematic uncertainties are calculated according to \cite{CMS-PAS-TOP-13-011}. The specifics for each source are given below and the results are summarized. Generally these uncertainties are obtained by changing a certain parameter, then rerunning the whole analysis and taking the deviation of the result as systematic uncertainty. The results are found in Table \ref{tab_ttg_sys}.

\begin{itemize}
\item \textbf{Pileup}: The uncertainty on the pileup distribution mainly comes fro the uncertainty on the overall luminosity and the inelastic proton-proton cross section. This analysis assumes approximately 20 additional proton-proton interactions per event, based on a proton-proton cross section of \SI{69.4}{\milli \barn} \cite{Antchev:2011vs}. The systematic uncertainty is calculated according to the uncertainty of the cross section of \SI{5.9}{\percent}.  
\item \textbf{Out of time pileup}: The uncertainty due to out of time pileup is estimated by mixing pseudo data from the out of time pile up and then fitting it with the templates used in the template fit. The difference between the fit results and the factors aplied in the mixing are then used as systematic uncertainty.
\item \textbf{Signal normalization}: This uncertainty is calculated by varying the signal yield by \SI{\pm 25}{\percent}. This includes the variation of the kinematic scales for the cross section calculation \cite{Melnikov:2011ta}.
\item \textbf{Background normalization}: The background normalization is varied within the individual cross section uncertainties. The single deviations are then summed up quadratically.
\item \textbf{Hadronization/Showering}: The impact of the model used for the simulation is estimated by comparing samples simulated with POWHEG \cite{Alioli:2011as} \cite{Nason:2004rx} \cite{Frixione:2007vw} \cite{Alioli:2010xd} and then showered and hadronized with either PYTHIA \cite{Sjostrand:2006za} or HERWIG \cite{Corcella:2000bw}.
\item \textbf{Generator}: This uncertainty is evaluated by comparing POWHEG and MADGRAPH samples showered with PYTHIA and POWHEG and MC@NLO \cite{Frixione:2002ik} \cite{Frixione:2003ei} samples showered with HERWIG. The largest deviation is taken as systematic uncertainty.
\item \textbf{PDF uncertainties}: Due to technical reasons they were not calculated for this analysis. In the previous analysis their impact was negligible \cite{CMS-PAS-TOP-13-011}.
\item \textbf{Top quark \pt}: There are differences between the \pt distribution of top quarks in data and simulation. Therefore simulated samples are reweighted according to the data distribution. In order to evaluate the uncertainty caused by the reweighing, the weights are either doubled or set to zero.
\item \textbf{Identification of b-jets}: The impact of b-tagging has to be corrected in simulation depending on the \pt , $\eta$ and flavour of a jet \cite{CMS-PAS-BTV-13-001}. The b-tagging uncertainty is estimated by varying these scaling factors within their uncertainties.
\item \textbf{Jet energy correction}: The jet energy correction factors are varied within their uncertainty.
\item \textbf{Jet energy resolution}: The jet energy resolution is smeared in simulation to match the resolution found in data. The smearing factors are varied within their uncertainties.
\end{itemize}

\begin{table}[ht]
\centering
\caption{Uncertainties on the relation between \ttbar and \ttgamma cross sections $R^{vis}$ and the \ttgamma cross section $\sigma_{\ttgamma}$.}
\label{tab_ttg_sys}
\begin{tabular}{l c c} \\

\hline

\hline

Source & \multicolumn{2}{c}{Uncertainty (\%)} \\

& $R^{vis}$ & $\sigma_{\ttgamma}$ \\

\hline

Statistical & 4.1 & 4.1 \\

\hline

Systematic & 18.2 & 19.4 \\

\hline

Individual contributions: & & \\

\;\;\;pileup & 0.4 & 0.3 \\

\;\;\;out-of-time pileup & 8.2 & 8.2 \\

\;\;\;$\sigma_{\mathrm{ sig}}$ & 9.0 & 9.0 \\

\;\;\;$\sigma_{\mathrm {bkg}}$ & 1.4 & 1.4 \\

\;\;\;shower./hadr. & 3.0 & 3.0 \\

\;\;\;generator & 13.0 & 13.0 \\

\;\;\;top-quark \pt & 0.5 & 0.3 \\

\;\;\;b-tag & 0.3 & 0.3 \\

\;\;\;JEC & 1.9 & 1.7 \\

\;\;\;JER & 0.3 & 0.4 \\

\hline

\textbf{Total} & 18.7 & 19.8 \\

\hline

\hline

\end{tabular}
\end{table}

\section{The \ttgamma Cross Section}

The ratio between the \ttbar and the \ttgamma cross section is determined according to Equation \ref{eq_ttg_cros_R}.

\begin{equation}
R \equiv \frac{R^{vis}}{\epsilon^{vis}_{\gamma}} = \frac{\sigma_{\ttgamma}}{1} \cdot \frac{1}{\sigma_{\mathrm{t}\overline{\mathrm{t}}}} = \frac{N^{sig}_{\ttgamma}}{\epsilon^{vis}_{\gamma} \epsilon_{\gamma}} \cdot \frac{1}{N^{presel} \pi_{\mathrm{t}\overline{\mathrm{t}}}}
\label{eq_ttg_cros_R}
\end{equation}

Here the purity of the \ttbar selection (see Equation \ref{eq_ttg_cros_purtt})   and the efficiency of the photon selection (see Equation \ref{eq_ttg_cros_effgam}) are determined from simulation.

\begin{equation}
\pi_\ttbar = N_\ttbar^{presel} / N^{presel} = \SI{84.3}{\percent}
\label{eq_ttg_cros_purtt}
\end{equation}

\begin{equation}
\epsilon_{\gamma}^{vis} \epsilon_{\gamma} = N_\ttgamma^{sel} / N_ttgamma^{presel} = \SI{62.9}{\percent}
 \label{eq_ttg_cros_effgam}
\end{equation}

The luminosity, detector acceptance and the preselection efficiency cancel out. The result is shown in Equation \ref{eq_ttg_cros_Rres}.

\begin{equation}
R^{vis} = (0.93 \;\pm 0.04\,\mathrm{stat}\;\pm 0.17\,\mathrm{syst})\cdot 10^{-2} \hspace{1cm} R = (1.15 \;\pm 0.05\,\mathrm{stat}\;\pm 0.21\,\mathrm{syst})\cdot 10^{-2}
\label{eq_ttg_cros_Rres}
\end{equation}

The cross section of the \ttgamma process is calculated by multiplying $R$ with $\sigma_\ttbar$. The CMS result is shown in Equation \ref{CMS-PAS-TOP-12-007} \todo{Better Value ?}.

\begin{equation}
\sigma_\ttbar^{\mathrm{CMS}} = \SI{227 \pm 15}{\pico \barn}
\label{eq_ttg_cros_restt}
\end{equation}

This leads to a the result for the \ttgamma cross section shown in Equation \ref{eq_ttg_cros_res}.

\begin{equation}
\sigma_{\ttgamma} \;=\; R\;\cdot \;\sigma_{\ttbar} \; =\; 2.6\;\pm \;0.1\mathrm{ (stat.)} \;\pm \;0.5\mathrm{ (syst.)}\; \SI{}{\pico \barn}
\label{eq_ttg_cros_restt}
\end{equation}

This is consistent with both the theoretical prediction of $\sigma_\ttgamma = \SI{1.8 \pm 0.5}{\pico \barn}$ and the CMS result of $\sigma_\ttgamma^{CMS} = \SI{2.4\pm 0.6}{\pico \barn}$.

\bibliography{Masterarbeit}

\end{document}