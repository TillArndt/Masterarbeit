\chapter{Simulation and Reconstruction}

In this chapter, the simulation and reconstruction of events in the CMS framework is described. The reconstruction is applied on both measured and simulated events.

\section{Simulated Events}
\label{sec_reco_simu}

The first simulation step consists of calculating the cross section of a certain process for the relevant phase space. Then events are randomly generated in this phase space. The simulated samples in this analysis, use the matrix-element generator MADGRAPH for this step \cite{Alwall:2014hca} with the CTEQ6L1 PDF \cite{Pumplin:2002vw} (unless otherwise noted). The single top samples are generated with POWHEG \cite{Nason:2004rx,Frixione:2007vw,Alioli:2010xd,Re:2010bp,Alioli:2009je}. \\
In the next step, the the decay of $\tau$ leptons is considered with TAUOLA \cite{Davidson:2010rw}. \\
Radiation from the initial and final state particle  as well as hadronization and showering are modeled with PYTHIA \cite{Sjostrand:2006za}. \\
The MLM matching procedure is applied in order to avoid ambiguities between matrix element generation and showering \cite{Mangano:2006rw}.\\
The underlying event is modeled with the Z2 tune \cite{Field:2012jv}. A number of soft interactions is added to the event, in order to estimate the effect of pile-up events. \\
The response of the detector is simulated with GEANT4 \cite{Agostinelli:2002hh}.

\section{Reconstruction}

The reconstruction implemented in the CMSSW framework is described here. \\
In the so called RECO step, a list of particle candidates is produced. Then the particle flow algorithm forms a list of unambiguous particle objects. Subsequently, combined objects like jets are created. As shown in Figure \ref{fig_reco_cms}, information from all detector subsystems is combined for the reconstruction. \\
After a short introduction of the tracking and supercluster algorithms, this section describes particle flow algorithm. At last the jet algorithm is outlined.\\
\newpage
\begin{figure}[ht]
    \centering
    \includegraphics[width = 0.8\textwidth] {Bilder/CMS_Slice}
  \caption{Slice of the CMS detector. Included is a illustration of particles passing through the CMS detector \cite{cms_slice}.}
  \label{fig_reco_cms}
\end{figure}

\subsubsection{The Tracking Algorithm}

The CMS tracking algorithm utilizes an iterative approach using Kalman-filter \cite{Fruhwirth:1987fm} techniques. Depending on the goodness of a fit and additional quality criteria, trackerhits are assigned to a track. The hits of accepted tracks are then removed from the hit collection for the next iteration. Primary and secondary vertices are reconstructed using these tracks. Details can be found in \cite{Adam:934067}.

\subsubsection{The Supercluster Algorithm}
\label{sec_reco_supcl}

In the ECAL, superclusters are used to estimate the impact of photon conversion and bremsstrahlung. Both effects lead to a spread in the energy deposition in the ECAL. The supercluster should include the energy of the particles produced in the shower, so that the original particle can be reconstructed. \\
In the barrel, a "hybrid" algorithm uses a fixed number of five pixels in $\eta$ direction and a variable number of clusters in $\phi$ direction.\\
In the endcaps, the "Multi5x5" algorithm uses clusters of a fixed size of 5x5 pixels around the highest energy deposition. \\
A more detailed description can be found in \cite{cmsTDR1}.

\subsection{The Particle Flow Algorithm}

The particle flow algorithm \cite{CMS-PAS-PFT-09-001} resolves ambiguities and produces a list of reconstructed particles. The input consists of tracks, clusters from the calorimeters and particle candidates. The steps are as follows:

\begin{itemize}
\item The inner tracks are linked to the calorimeter clusters and tracks in the muon system. The quality of the link is determined by spatial distance (calorimeter clusters) or a fit (muon system). Bremsstrahlung is included and the link with the best quality is taken.
\item Tracks with that link to the muon system are identified as muons and removed from the collection. For each muon, an average amount of energy is subtracted from the calorimeter clusters.
\item Electron candidates are re-investigated by fitting the track to a calorimeter entry including Bremsstrahlung. If the candidate is accepted as an electron, the calorimeter entries and the track are removed.
\item Remaining tracks are considered to be charged hadrons. If a matching calorimeter cluster can be found, the tracks are refitted.
\item Any surplus of energy in the calorimeters is considered to be either a photon or a neutral hadron. If the energy deposition is primarily in the ECAL, a photon is reconstructed. If it is primarily in the HCAL, the particle is considered to be a neutral hadron.
\end{itemize}

\subsection{The Reconstruction of a Jet}
\label{sec_reco_jet}

A jet is a combination of collinear particles within a certain area. It is mostly considered to arise from a single parton or gluon due to hadronization and the subsequent decay of the hadron and / or showering. \\
Jet algorithms are used to cluster particles that are believed to come from the same parton. The sum of the four momenta of all particles in the jet (the four momentum of the jet) is assigned to the four momentum of the original particle. \\
These jet algorithms should be stable in respect to collinear emission, where two particles might be reconstructed as one, and soft emission, which might not be detected. \\
In this analysis, the anti-kt algorithm is used \cite{Cacciari:2008gp}, applying a distance parameter of $\rho = 0.5$. Particle flow objects are used for the clustering. The jet energy has to be corrected, because of detector effects and pile-up. The uncertainties of this corrections (see Figure \ref{fig_reco_jet} ) are a good indicator for the performance of the jet algorithm.\\

\begin{figure}[ht]
   \subfigure[]{
    \includegraphics[width = 0.4\textwidth] {Bilder/AK5PF_Eta00}
  }
    \subfigure[]{
    \includegraphics[width = 0.4\textwidth] {Bilder/AK5PF_Eta27}
 }
  \caption{Uncertainties on the jet energy corrections for (a) $\eta = 0$ and (b) $|\eta|=2.7$ \cite{Jet_Perf}.}
  \label{fig_reco_jet}
\end{figure}

\newpage
\subsubsection{B-Tagging}
\label{sec_reco_jet_btag}

The identification of the jet flavor, i.e. the flavor of the original quark resulting in the jet, is difficult. Nevertheless, it is possible to do so for b quarks due to the long lifetime of B mesons. In CMS these b-jets typically originate from a secondary vertex corresponding to the decay of the original B-meson. Because of the precision of the inner tracking system ,this secondary vertex can be separated from the primary vertex of the hard interaction. \\
In this analysis, a combined secondary vertex b-tag (CSVM) \cite{Chatrchyan:2012jua} at medium operation point is used. It employs a multivariate analysis with track based observables to distinguish b-jets from light-flavor jets. In Figure \ref{fig_reco_btag} the performance of the CSV algorithm is shown. The exact number applying to this analysis are given in Section \ref{sec_ttg_presel}. \\

\begin{figure}[ht]
\centering
   \subfigure[]{
   \includegraphics[width = 0.38\textwidth] {Bilder/CSV_btag}
  } \\
    \subfigure[]{
    \includegraphics[width = 0.47\textwidth] {Bilder/CSV_btag_eff}
 }
     \subfigure[]{
    \includegraphics[width = 0.47\textwidth] {Bilder/CSV_btag_miss}
 }
  \caption{(a) Result of the CSV algorithm in data and simulation. (b) Efficiency of the CSV algorithm depending on the working point in data and simulation. (c) Misidentification rate of the CSV algorithm at medium working point depending on the /pt in data and simulation. \cite{CMS-PAS-BTV-13-001}}
  \label{fig_reco_btag}
\end{figure}