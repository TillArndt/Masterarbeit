\chapter{The LHC and the CMS detector}

This chapter shortly describes the Large Hadron Collider. In the second part the CMS detector and its most important subsystems are explained.

\section{ The Large Hadron Collider}

The Large Hadron Collider (LHC) \cite{LHCTDR} is located in a tunnel beneath the French Swiss border region near Geneva. The tunnel lies about \SI{100}{\meter} below the ground with a cicumference of \SI{26.7}{ \kilo \meter}. The LHC uses 1232 superconducting magnets to produce a magnetic field with the maximal strength with a of \SI{8.3}{\tesla}.\\
The LHC can accelerate protons as well as heavy ions. For the proton-proton collision two counter rotating beams of protons are accelerated and then brought together at the four interaction points. According to design the proton collisions are provided with a center of mass energy of $\sqrt{s} = \SI{14 }{\tera \electronvolt}$ and a peak luminosity of $\mathcal{L} = \SI{d34}{\per \centi \meter \tothe{2} \per \second}$. \\
The accelerator complex is shown in Figure \ref{fig_det_accel}. In order to reach high energies protons are preaccelerated by the Linear Accelerator 2 (Linac2), the Proton Synchrotron Booster (PSB), the Proton Synchrotron (PS) and finally by the Super Proton Synchrotron (PSP). Through the preacceleration the protons reach a energy of $\SI{450}{\giga \electronvolt}$ for injection into the LHC. \\


\begin{figure}[ht]
    \centering
    \includegraphics[width = 0.7\textwidth] {Bilder/LHC_accel}
  \caption{The Cern accelerator complex. \cite{Lefevre:1165534} The pre accelerators and the four main LHC detectors are shown. Other experiments participate as well.}
  \label{fig_det_accel}
\end{figure}

Four major detectors are located around the LHC ring. \\
CMS \cite{cmsTDR1} and ATLAS \cite{AtlasTDR} are multipurpose detectors. They are designed to cover a large range of physics topics and particles. Additionally, each should be able to corroborate or refute the results of the other experiment allowing for independent validation of these results.\\
ALICE \cite{AliceTDR} is a detector specialized for heavy ion physics. One of the central subjects is the investigation of quark-gluon plasma, possibly created in high energy heavy ion collisions. ALICE is able to record and reconstruct events with a very high particle multiplicity (about 50,000 particles per event). \\
LHCb \cite{LHCbTDR} is only measuring particles on one side of the interaction point. The detector is build for b-physics concentrrating on precision measurements of the standard model for example in the CP-violating sector.\\
LHCf \cite{LHCfTDR} and TOTEM \cite{TOTEMTDR} are situated in the forward region of of ATLAS and CMS respectively. The TOTEM experiment is focused on the proton substructure. LHCf is concerned with development of hadronic showers and the associated phenomenological methods. \\
In 2012 the beam energy was set to \SI{4}{\tera \electronvolt} ($\sqrt{s} = \SI{14}{\tera \electronvolt}$). CMS recorded a luminosity of $\mathcal{L} = \SI{21.79}{\per \femto \barn}$ of the \SI{23.30}{\per \femto \barn} provided by the LHC. \\

\begin{figure}[ht]
  \subfigure[]{
    \includegraphics[width = 0.55\textwidth] {Bilder/lumi_cms_2012}
      }
    \subfigure[]{
    \includegraphics[width = 0.49\textwidth] {Bilder/lumi_lhc_2012}
  }
  \caption{(a): Luminosity delivered by the LHC and recorded by CMS in 2012.\cite{lumi_cms} (b): Luminosity delivered to the four major experiments at the LHC in 2012.\cite{lumi_lhc}}
  \label{fig_det_lumi}
\end{figure}

After the current shutdown of LHC ends in 2015 the center of mass energy will be raised to $\sqrt{s} = \SI{13}{\tera \electronvolt}$.

\section{The Compact Muon Solenoid}

\begin{figure}[ht]
    \includegraphics[width = 0.75\textwidth] {Bilder/cms_complete_labelled}
      
  \caption{Vertical section of the whole cms detector. \cite{CMS_Draw}}
  \label{fig_cms_draw}
\end{figure}