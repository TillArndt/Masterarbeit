\chapter{An improved Template Fit for the \ttgamma Cross Section Measurement}
\label{sec_ttg}

In this section an improved measurement of the \ttgamma cross section is presented. It is based on the analyses presented in \cite{CMS-PAS-TOP-13-011} and \cite{tholen:ma} and takes a closer look at certain aspects. The major change is a revised template fit for the determination of the final cross section.\\
As described in Section \ref{sec_simu_comp_2to7} a new factorized $2 \to 7$ has been developed for this measurement. However, the full simulation was not ready in time to be included here. Because the full official simulation has been performed by the CMS collaboration, the timescale is beyond the author's control. Consequently, the $2 \to 5$ approach from \cite{CMS-PAS-TOP-13-011} described in Section \ref{sec_simu_comp_2to5} is used for the \ttgamma cross section measurement.  \\

\section{Outline of the \ttgamma Cross Section Measurement}


The \ttgamma cross section measurement is performed following the steps below (see also Figure \ref{fig_ttg_out_diag}):\\

\begin{description}
\item[Simulation]: Signal (\ttgamma) events are simulated including showering, hadronization and simulation of detector response (as described in Section \ref{sec_reco_simu}). The $2 \to 5$ simulation strategy is used. The simulated background events are provided by official CMS samples.
\item[Preselection]: To select semi muonic top-quark pair events (see Section \ref{sec_ttg_strat_sig} for the definition of the signal process), a CMS reference selection (see Section \ref{sec_ttg_presel}) is used. The number of preselected events in data $N^{presel}$ and the purity $\pi_{\mathrm{t}\overline{\mathrm{t}}}$ of the selection are calculated from simulated events.
\item[Selection]: Events with a prompt photon are selected (see Section \ref{sec_ttg_sel}). The efficiency $\epsilon_{\gamma}$ is computed from simulation.
\item[Fake photon rate]: A template fit extracts the number of real photons in the final candidate collection of prompt photons. Templates for real and fake photons are taken from data and fitted to a shower shape distribution (see Section \ref{sec_ttg_fit}). The result of the fit is the number of selected \ttgamma events in data $N^{sig}_{\ttgamma}$.
\item[Closure Test]: The validity of the fit procedure is analyzed with closure tests (see Section \ref{sec_ttg_clo}). Pseudo data is generated from simulated samples. Afterwards, the template fit is applied to the pseudo data. The fit results should reproduce the input of the pseudo-data generation.
\item[R-ratio]: The ratio between the \ttgamma and the $\mathrm{t}\overline{\mathrm{t}}$ cross section is calculated (see Equation \ref{eq_ttg_out_R}):

\begin{equation}
R \equiv \frac{R^{vis}}{\epsilon^{vis}_{\gamma}} = \frac{\sigma_{\ttgamma}}{1} \cdot \frac{1}{\sigma_{\mathrm{t}\overline{\mathrm{t}}}} = \frac{N^{sig}_{\ttgamma}}{\epsilon^{vis}_{\gamma} \epsilon_{\gamma}} \cdot \frac{1}{N^{presel} \pi_{\mathrm{t}\overline{\mathrm{t}}}}.
\label{eq_ttg_out_R}
\end{equation}

$\epsilon_{\gamma}^{vis}$ is the efficiency of the phase-space restriction in the analysis. The inverse $1 / \epsilon_{\gamma}^{vis}$ corrects the measured result in the fiducial volume to correspond to the physically relevant $R$. Other quantities, like the luminosity or the efficiency of the preselection, cancel out. 
\item[Result] After discussing systematic uncertainties (see Section \ref{sec_ttg_sys}), the \ttgamma cross section is computed from the measured top-quark pair cross section (see Section \ref{sec_ttg_crossec}, Equation \ref{eq_ttg_out_cross}):

\begin{equation}
\sigma_{\ttgamma} = R \cdot \sigma_{\mathrm{t}\overline{\mathrm{t}}}.
\label{eq_ttg_out_cross}
\end{equation}

\end{description}

\begin{figure}[hb]
\centering
    \includegraphics[width = .85\textwidth] {Bilder/Complete_Flow}
  \caption{Flow chart showing the structure of the analysis.}
  \label{fig_ttg_out_diag}
\end{figure}

\section{Background Processes}

After describing the signal definition in Section \ref{sec_ttg_strat_sig}, this section describes the background processes that are able to fake such signatures.  
These background processes can be divided into two parts. On the one hand there are processes which have a top-quark pair like signature. On the other hand there are non-signal type photons or photon like particles in a top-quark pair event.\\

\subsection{Top-Quark Pair like Background}

In order to have a top-quark pair like signature events should mainly have one muon, missing transverse energy and four high energetic jets. The most important background processes are shown in Figure \ref{fig_ttg_bkg} and discussed below.

\begin{itemize}
\item \textbf{W + Jets}: The signature consists of a muon and missing transverse energy from the W-boson decay, additional jets can be produced through QCD radiation (see Figure\ref{fig_ttg_bkg_wjets}). Requiring additional b-jets reduces this background. Moreover, the jets typically have a lower energy than the ones originating from top-quark pair events.
\item \textbf{Drell-Yan + Jets}: A photon or a Z boson decays into two muons and additional jets are produced (see Figure \ref{fig_ttg_bkg_dy}). One of the muons must be very low energetic or misidentified, otherwise the muon veto (see Section \ref{sec_ttg_presel} for the preselection) would reject the event. Missing transverse energy can be caused by particles not being detected or by mismeasured momenta. Tight requirements on the jets again reduce the cross section of this process.
\item \textbf{Single Top}: Single top-quark are produced either in the s-channel, the t-channel or associated with a W boson (see Figure \ref{fig_ttg_bkg_singlet} for the s-channel). Even though the cross sections are small, the very signal like structure allows for a significant contribution to the signal region.
\item \textbf{QCD Multijet}: This background is always a concern when dealing with interactions at a hadron collider because of its very large cross section. It is unlikely that a jet could be reconstructed as a muon, making the process less relevant for the signal region (see Figure \ref{fig_ttg_bkg_qcd}).
\item \textbf{Other top-quark pair decays}: Other decay channels for top-quark pairs can also contribute in the signal region, due to their similar structure. Especially events where a tau lepton decays into a muon contribute. 
\end{itemize}

\begin{figure}[ht]
  \subfigure[W + Jets]{
    \includegraphics[width = 0.52\textwidth] {Bilder/feyn_wjets}
    \label{fig_ttg_bkg_wjets}
  }
    \subfigure[Drell-Yan + Jets]{
    \includegraphics[width = 0.52\textwidth] {Bilder/feyn_dy}
        \label{fig_ttg_bkg_dy}
 }
 \\
   \subfigure[Single Top (s-channel)]{
    \includegraphics[width = 0.52\textwidth] {Bilder/feyn_singlet}
        \label{fig_ttg_bkg_singlet}
  }
    \subfigure[QCD]{
    \includegraphics[width = 0.52\textwidth] {Bilder/feyn_qcd}
        \label{fig_ttg_bkg_qcd}
 }
  \caption{Example Feynman graphs for the \ttbar-like background events. Additional jets are generated through quark or gluon radiation \cite{steeger}.}
  \label{fig_ttg_bkg}
\end{figure}

\FloatBarrier
\subsection{Photon Signature Background}


There are two different types of photon backgrounds. One type of background events contain no physical photon. Here, the photon is faked from another particle like a neutral hadron, an electrons, a photon from a neutral pion or another particle. The photons from the neutral pions are actual photons, but they are not considered to be final state photons. These mis-reconstructed particles are called fake photons throughout the analysis. \\
The second type of background contains real photons that are not radiated from the top quark, but from initial state or final state particles (see Figures \ref{fig_ttg_bkg_fsr} and \ref{fig_ttg_bkg_isr}). Depending on the signal definition and the simulation strategy (see Section \ref{ch_simu_con} for the simulation strategies) some of these events might actually belong to the signal region. For this measurement the $2 \to 5$ strategy is used for the signal simulation, so only photons radiated from the light quarks or the muon (see Figures \ref{fig_ttg_bkg_fsrf1} and \ref{fig_ttg_bkg_fsrf2}) are considered to contribute to the background region. Nevertheless, the selection (see Section \ref{sec_ttg_sel}) aims to mitigate the influence of the modeling of initial and final state radiation.

\begin{figure}[ht]
  \subfigure[Quark Annihilation]{
    \includegraphics[width = 0.5\textwidth] {Bilder/feyn_ttgamma_isrqq}
    \label{fig_ttg_bkg_isrqq}
  }
    \subfigure[Gluon Fusion]{
    \includegraphics[width = 0.5\textwidth] {Bilder/feyn_ttgamma_isrgg}
        \label{fig_ttg_bkg_isrgg}
 }
  \caption{Background by additional photon radiation in \ttbar events. Here the photons are radiated from the initial state particles. Only quarks (a) in the initial state can radiate photons \cite{Hermanns:1292768}.}
  \label{fig_ttg_bkg_isr}
\end{figure}

\begin{figure}[ht]
  \subfigure[]{
    \includegraphics[width = 0.5\textwidth] {Bilder/fey_ttgamma_fsrb}
    \label{fig_ttg_bkg_fsrb}
  }
    \subfigure[]{
    \includegraphics[width = 0.5\textwidth] {Bilder/feyn_ttgamma_fsrw}
        \label{fig_ttg_bkg_fsrw}
 }
 \\
   \subfigure[]{
    \includegraphics[width = 0.5\textwidth] {Bilder/fey_ttgamma_fsrf1}
        \label{fig_ttg_bkg_fsrf1}
  }
    \subfigure[]{
    \includegraphics[width = 0.5\textwidth] {Bilder/fey_ttgamma_fsrf2}
        \label{fig_ttg_bkg_fsrf2}
 }
  \caption{Background by additional photon radiation in \ttbar events. Here, the photon is radiated from one of the charged final state or intermediary particles \cite{tholen:ma,Hermanns:1292768}.}
  \label{fig_ttg_bkg_fsr}
\end{figure}



\section{Preselection: Top-Quark Pair Events}
\label{sec_ttg_presel}

The preselection is based on CMS recommendations for a cut based top-quark pair event selection. A similar selection is implemented in \cite{CMS-PAS-TOP-12-027}. Here, the selection requires exactly one high energetic muon, four high energetic jets and at least one of them should be b-tagged. The details are found below.

\begin{description}
\item[High Level Trigger]: All events are required to have passed the \text{HLT\_IsoMu24\_2p1} trigger path. It requires an isolated muon with a transverse momentum of $\pt > \SI{24}{\giga \electronvolt}$ and $| \eta | < 2.1$. This trigger is not prescaled.
\item[Muon]: The events must have exactly one high energetic and isolated muon. The muons are required to have  $\pt > \SI{26}{\giga \electronvolt}$ to reach the plateau of the turn-on curve of the trigger efficiency and a pseudo rapidity $| \eta | <2.1 $. They are required to be global muons, meaning they have to be independently reconstructed in the tracker as well as muon system. The relative isolation should be $I_{rel}(0.4) < 0.12$. Here, $I_{rel}$ is defined as the sum of the \pt of all neutral and charged particle candidates in a cone of $\Delta R < 0.4$ . This sum is divided by the \pt of the muon and the isolation is corrected for pile up. The distance from the primary vertex is required to be $\Delta z < \SI{5}{\milli \meter}$ and $\Delta \rho < \SI{0.2}{\milli \meter}$. The combined efficiency of the muon trigger and selection is estimated to be $\SI{76}{\percent}-\SI{94}{\percent}$ contingent on the $\eta$ and \pt of the muon \cite{CMS-DP-2013-009,Chatrchyan:2012xi}.
\item[Primary Vertex]: All events should have at least one good primary vertex: The distance to the beam interaction region should be $\Delta z < \SI{24}{\centi \meter}$ and $\Delta \rho < \SI{2}{\centi \meter}$. If there are multiple vertices the one with the highest $\pt^2$ of all associated tracks is chosen as primary vertex. All leptons are required to originate from the primary vertex.
\item[Jets]: The events should have at least four jets with $| \eta | < 2.5$. The transverse momenta of the four leading jets should be larger than \SI{55}{\giga \electronvolt}, \SI{45}{\giga \electronvolt}, \SI{35}{\giga \electronvolt} and \SI{20}{\giga \electronvolt} respectively. 
\item[b-tag]: At least one of the four leading jets must be b-tagged (see Section \ref{sec_reco_jet_btag} for a short description of the b-tag algorithm). The combined secondary vertex algorithm with a value of CVSM$\;> 0.679$ is used. This working point has an efficiency of approximately \SI{70}{\percent} and a misidentification rate of approximately \SI{1.5}{\percent} \cite{CMS-PAS-TOP-12-027,Chatrchyan:2012jua}.
\item[Electron Veto]: Events with an electron candidate with $\pt > \SI{20}{\giga \electronvolt}$, $| \eta |< 2.5$ and $I_{rel}(0.3)< 0.15$ are rejected.
\item[Loose Muon Veto]: All events with an additional global muon with $\pt > \SI{10}{\giga \electronvolt}$, $|\eta < 2.5|$ and $I_{rel}(0.4)<0.20$ are rejected. 
\end{description}

No requirement on \ETm is applied, because of the jet energy scale leading to large uncertainties on \ETm .\\
Single top quark and W + jets events are the main background contributions. The number of preselected events in data is $N^{presel} =256665$. The fraction of \ttbar events in the \ttgamma signal region is determined from simulation $\pi_{\ttbar} = N^{presel}_{\ttbar} / N^{presel} = \SI{84.3}{\percent}$. The simulated samples are normalized to data luminosity. About \SI{9}{\percent} of W + jets and \SI{5}{\percent} of single top quark events make up the rest of the preselected events.\\

\section{Photon Selection}
\label{sec_ttg_sel}

The photon candidates are reconstructed using significant energy deposits in the ECAL superclusters. Several isolation and kinematic variables are used for the cut based selection, details are given below. The effect of the underlying event and pile-up on these isolation criteria are treated with the so called $\rho$-correction \cite{CMS-PAS-PFT-09-001}. The selection is based on the loose cut-based photon identification \cite{CMS-PAS-HIG-13-006}. In order to pass the selection cuts, the events are required to have at least one photon that fulfills the conditions.

\begin{description}
\item[Fiducialization]: The photon candidate is required to have a transverse energy $E_{\mathrm{T}} > \SI{25}{\giga \electronvolt}$ in order so suppress low energy fakes and photons not originating from the primary vertex. There is an additional cut of $| \eta | < 1.4442$ to restrict the photons to the barrel region of the ECAL. The photon identification in the endcaps is not reliable due to the large material budget in front of it \cite{tholen:ma}.
\item[Electron veto]: The photon candidate is required to have no track seed in the pixel detector. Due to the electron photon disambiguation employed in the particle reconstruction all photon candidates pass this cut.
\item[Tower-based H/E]: The ratio between the energy deposition in the HCAL towers and the respective ECAL towers should be less than \SI{5}{\percent}.
\item[Shower width $\sigma_{i \eta i \eta}$]: The variable $\sigma_{i \eta i \eta}$ describes the shape and width of the energy deposition in the ECAL (see Equation \ref{eq_ttg_sel_sie}). The requirement is $\sigma_{i \eta i \eta} < 0.012$.
\begin{equation}
\sigma_{i \eta i \eta} = (\frac{\sum (\eta_i - \bar{\eta})^2 \omega_i}{\sum \omega_i})^{1/2} ; \hspace{0.4cm} \bar{\eta} = \frac{\sum \eta_i \omega_i}{\sum \omega_i}; \hspace{0.4cm} \omega_i = max(0,4.7 + \log{\frac{E_i}{E_{5 \times 5}}})
\label{eq_ttg_sel_sie}
\end{equation}
Here $E_{5 \times 5}$ is the energy in the previously mentioned supercluster of the photon (see Section \ref{sec_reco_supcl} for details on the superclusters).
\item[Neutral hadron isolation]: The sum of the \pt of neutral hadron candidates in a cone of $\Delta \mathrm{R} < 0.4$ around the photon. The requirement is $I_{neu.had.} < \SI{3.5}{\giga \electronvolt} + 0.04 \cdot E_{\mathrm{T}}(\gamma)$.
\item[Photon isolation]: The sum of the \pt of other photon candidates around ($\Delta \mathrm{R} < 0.4$) the photon. The requirement is $I_{photon} < \SI{1.3}{\giga \electronvolt} + 0.005 \cdot E_{\mathrm{T}}(\gamma)$. 
\item[FSR suppression]: In order to suppress final state radiation the photons are required to have a minimal distance from the final state particles. The condition is $\Delta R (\gamma,\mu /j) > 0.7$.
\end{description}

As seen in Table \ref{tab_ttg_sel_evt}, data and simulation are in agreement after the selection. The signal to background ratio is $S/B = 0.14$ and the \ttgamma selection efficiency is $\epsilon_\gamma = N^{sel}_{\ttgamma} / N^{presel}_{\ttgamma} = \SI{62.9}{\percent}$. The total number of selected data events is $N^{sel} = 7836$. The performance of the selection is satisfactory and the number of events is sufficient to perform a template fit.

\begin{landscape}
\begin{table}
\caption{Cut flow table: Number of events after the corresponding selection steps of the prompt photon selection for different event samples. The uncertainties given here are statistical errors only.}
\begin{tabular}{l | r r r r r r r r r r }

                  &           presel. &           $E_{T}$ &          $|\eta|$ &      $e^\pm$ veto &             H / E & $\sigma_{i\eta i\eta}$ &                      neutr. iso &          pho. iso & $\Delta R(\gamma, \mu)$ & $\Delta R(\gamma, j)$ \\

\hline

\hline

         \ttgamma (signal) &$            1872.8 $&$            1737.3 $&$            1508.7 $&$            1508.7 $&$            1444.5 $&$            1313.2 $&$            1271.5 $&$            1050.3 $&$             983.0 $&$             949.4 $ \\

                  &$ \pm              8.1 $&$ \pm              7.8 $&$ \pm              7.3 $&$ \pm              7.3 $&$ \pm              7.1 $&$ \pm              6.8 $&$ \pm              6.7 $&$ \pm              6.1 $&$ \pm              5.9 $&$ \pm              5.8 $ \\

             Single top &$            2275.4 $&$            1258.0 $&$             773.2 $&$             773.2 $&$             621.3 $&$             132.8 $&$             110.9 $&$              37.5 $&$              25.8 $&$              25.8 $ \\

             (t-channel)&$ \pm            159.1 $&$ \pm            118.3 $&$ \pm             92.8 $&$ \pm             92.8 $&$ \pm             83.2 $&$ \pm             38.4 $&$ \pm             35.1 $&$ \pm             20.4 $&$ \pm             16.9 $&$ \pm             16.9 $ \\

            Single top  &$            4088.2 $&$            2148.2 $&$            1735.5 $&$            1735.5 $&$            1305.8 $&$             511.4 $&$             432.9 $&$             142.6 $&$             122.0 $&$             122.0 $ \\

            (W-associated)      &$ \pm             42.4 $&$ \pm             30.7 $&$ \pm             27.6 $&$ \pm             27.6 $&$ \pm             24.0 $&$ \pm             15.0 $&$ \pm             13.8 $&$ \pm              7.9 $&$ \pm              7.3 $&$ \pm              7.3 $ \\

          Single anti-top &$            1267.7 $&$             667.1 $&$             459.8 $&$             459.8 $&$             343.2 $&$             130.2 $&$             107.2 $&$              22.4 $&$              19.6 $&$              19.6 $ \\

          (t-channel)        &$ \pm             19.9 $&$ \pm             14.4 $&$ \pm             12.0 $&$ \pm             12.0 $&$ \pm             10.4 $&$ \pm              6.4 $&$ \pm              5.8 $&$ \pm              2.6 $&$ \pm              2.5 $&$ \pm              2.5 $ \\

         Single anti-top &$            4164.1 $&$            2148.8 $&$            1746.5 $&$            1746.5 $&$            1311.6 $&$             525.9 $&$             443.6 $&$             110.1 $&$              97.9 $&$              97.1 $ \\

           (W-associated)     &$ \pm             43.0 $&$ \pm             30.9 $&$ \pm             27.8 $&$ \pm             27.8 $&$ \pm             24.1 $&$ \pm             15.3 $&$ \pm             14.0 $&$ \pm              7.0 $&$ \pm              6.6 $&$ \pm              6.6 $ \\

          Drell-Yan + Jets &$            2961.0 $&$            1458.4 $&$            1058.4 $&$            1058.4 $&$             805.5 $&$             374.6 $&$             328.3 $&$             169.3 $&$             152.2 $&$             152.2 $ \\
          
                  &$ \pm             82.1 $&$ \pm             57.6 $&$ \pm             49.1 $&$ \pm             49.1 $&$ \pm             42.8 $&$ \pm             29.2 $&$ \pm             27.4 $&$ \pm             19.6 $&$ \pm             18.6 $&$ \pm             18.6 $ \\

            W + Jets &$           22662.1 $&$           10925.7 $&$            8300.0 $&$            8300.0 $&$            5809.2 $&$            2165.4 $&$            1689.9 $&$             489.3 $&$             406.9 $&$             406.9 $ \\

                  &$ \pm            538.8 $&$ \pm            374.1 $&$ \pm            326.1 $&$ \pm            326.1 $&$ \pm            272.8 $&$ \pm            166.6 $&$ \pm            147.1 $&$ \pm             79.2 $&$ \pm             72.2 $&$ \pm             72.2 $ \\

           \ttbar &$          205099.4 $&$          106788.6 $&$           86539.9 $&$           86539.9 $&$           68074.9 $&$           27888.3 $&$           23940.0 $&$            6521.1 $&$            6022.0 $&$            5986.4 $ \\

                  &$ \pm            386.7 $&$ \pm            279.0 $&$ \pm            251.2 $&$ \pm            251.2 $&$ \pm            222.8 $&$ \pm            142.6 $&$ \pm            132.1 $&$ \pm             68.9 $&$ \pm             66.3 $&$ \pm             66.1 $ \\

\hline

           MC Sum &$          244390.8 $&$          127132.0 $&$          102121.9 $&$          102121.9 $&$           79716.0 $&$           33042.0 $&$           28324.3 $&$            8542.5 $&$            7829.4 $&$            7759.4 $ \\

                  &$ \pm            690.0 $&$ \pm            487.2 $&$ \pm            426.8 $&$ \pm            426.8 $&$ \pm            366.2 $&$ \pm            225.7 $&$ \pm            203.8 $&$ \pm            109.5 $&$ \pm            101.9 $&$ \pm            101.7 $ \\

\hline

             RunA &$             11221 $&$              5848 $&$              4674 $&$              4674 $&$              3706 $&$              1523 $&$              1322 $&$               412 $&$               383 $&$               379 $ \\

             RunB &$             57067 $&$             29608 $&$             23538 $&$             23538 $&$             18598 $&$              7797 $&$              6704 $&$              1963 $&$              1782 $&$              1764 $ \\

             RunC &$             92718 $&$             48350 $&$             38650 $&$             38650 $&$             30711 $&$             12785 $&$             10986 $&$              3092 $&$              2827 $&$              2813 $ \\

             RunD &$             95659 $&$             49461 $&$             39548 $&$             39548 $&$             31412 $&$             12965 $&$             11168 $&$              3178 $&$              2901 $&$              2880 $ \\

\hline

         Data Sum &$            256665 $&$            133267 $&$            106410 $&$            106410 $&$             84427 $&$             35070 $&$             30180 $&$              8645 $&$              7893 $&$              7836 $ \\

\hline

\end{tabular}
\label{tab_ttg_sel_evt}
\end{table}
\end{landscape}

\section{Analysis of the Real Photon Content}

The cut-based identification does not sufficiently reduce the number of non-photon particles identified as a photon. Therefore, a template fit is applied in order to determine the number of events with signal photons. A new template fit is introduced that is completely data driven. \\
Templates for real and fake photons are taken from data and then fitted to the distribution of the charged hadron isolation. The template for fake photons is taken from a data sideband, whereas the template for real photons is obtained by using the so called random cone isolation.\\
The purity of the \ttgamma events is calculated from simulation.\\

\subsection{Photon Matching}

In simulation reconstructed photons can be assigned to generated photons. Moreover, it is possible to validate if the reconstructed photon corresponds to a real photon or if it is faked by another object. A photon that matches to a generated photon that is not part of a jet is classified as real photon. Photons that are not matched according to the above criteria are classified as fake photons.\\
The matching requires the generated and reconstructed photons to be within a cone of $\Delta \mathrm{R} < 0.2$ and the energy difference to be $\Delta E / E_{gen} < \SI{100}{\percent}$. The closest match by $\Delta \mathrm{R}$ is considered. The requirements are based on CMS standards and the performance of the matching is sufficient (see Figure \ref{fig_ttg_match}).

\begin{figure}[ht]
  \subfigure[]{
    \includegraphics[width = 0.49\textwidth] {Bilder/Match_dE}
  }
    \subfigure[]{
    \includegraphics[width = 0.49\textwidth] {Bilder/Match_dR}
 }
  \caption{Performance of the matching algorithm. (a) shows the relative energy differences between matched and generated photons, while (b) shows the difference in $\Delta \mathrm{R}$. The different samples are staked onto each other.}
  \label{fig_ttg_match}
\end{figure}

\subsection{The Template Fit}
\label{sec_ttg_fit}

The observable that is used for the discrimination between fake and real photons is described. Then, the way the templates for real and fake photons are obtained is explained. The result of the template fit is presented and discussed. At last, the performance of the template fit is tested with a closure test. \\

\subsubsection{The Discriminating Observable: Charged Hadron Isolation}

The charged hadron isolation is similar to the other particle based isolations described in Section \ref{sec_ttg_sel}. It is the sum of the \pt of all charged particle candidates in a cone of $\Delta R = 0.4$ around the photon. Therefore, it should be a good discriminator against jets misidentified as photons, which make up a large number of the faked photons \cite{tholen:ma}.\\  
Figure \ref{fig_ttg_fit_chhadnm1} shows the distribution of the charged hadron isolation at final cut level. The agreement between data and simulation is acceptable. There is a significant disagreement between data and simulation in the first, signal dominated bin. This is one of the reasons why a fully data driven fit should yield a more reliable result.\\
For the template fit only the photon with the highest transverse energy is considered for each event. Additionally, particles pointing to the supercluster of the photon are removed from the calculation of the isolation. This so called supercluster footprint removal is used to minimize the correlation between the charged hadron isolation and the shower shape observable $\sigma_{i \eta i \eta}$ \cite{CMS-PAS-HIG-13-006}.\\

\begin{figure}[hb]
\centering
    \includegraphics[width = 0.63\textwidth] {Bilder/Nm1_ChHadIso}
  \caption{Logarithmic plot of the charged hadron isolation after the selection. The simulated samples are normalized to data luminosity. The simulated samples are stacked.}
  \label{fig_ttg_fit_chhadnm1}
\end{figure}

\subsubsection{The Background Template}
\label{sec_ttg_fit_bkg}

The background template is taken from a data sideband (see \cite{CMS-PAS-TOP-13-011} for details). Two methods to obtain the sideband are used. For the first, three selection requirements are utilized. In this analysis, the method is called sideband ID (SBID). Photon candidates are selected for this sideband if they fail at least one of the three following conditions (see Section \ref{sec_ttg_sel} for a more in depth explanation of the selection, see Figure \ref{fig_ttg_nm1} for the observables): 
\begin{itemize}
\item $\sigma_{i \eta i \eta} > 0.012$,
\item $I_{neu.had.} > \SI{3.5}{\giga \electronvolt} + 0.04 \times E_{\mathrm{T}}(\gamma)$ or 
\item $I_{photon} > \SI{1.3}{\giga \electronvolt} + 0.005 \times E_{\mathrm{T}}(\gamma)$.
\end{itemize}

\begin{figure}[hb]
\centering
  \subfigure[]{
    \includegraphics[width = 0.4\textwidth] {Bilder/Nm1_SIEIE}
  }
    \subfigure[]{
    \includegraphics[width = 0.4\textwidth] {Bilder/Nm1_NeHadIso}
 }
     \subfigure[]{
    \includegraphics[width = 0.4\textwidth] {Bilder/Nm1_PhoIso}
 }
  \caption{ N-1 plots of the observables used for obtaining the side band distributions. The Samples are stacked onto each other. (a) shows $\sigma_{i \eta i \eta}$, (b) shows the neutral hadron isolation and (c) the photon isolation}
  \label{fig_ttg_nm1}
\end{figure}
\FloatBarrier
An alternative sideband distribution is used for comparison. It only has one requirement to have $0.012 < \sigma_{i \eta i \eta} < 0.018$. The number of photons in this sideband region is comparatively small.\\
The two sideband distributions are compared with the matched photon fake distribution from the simulated \ttbar sample (see Figure \ref{fig_ttg_fit_faketemplcomp}). The sideband distributions are also taken from the \ttbar sample. All three distributions are rebinned for the the template fit in order to mitigate the impact of low statistics. There are differences between the fake and the sideband distributions, which could be due to  mismodeling in the simulation. The reweighting of the templates that is used in \cite{CMS-PAS-TOP-13-011} is not applied here.

\begin{figure}[ht]
\centering
    \includegraphics[width = 0.67\textwidth] {Bilder/FakeTemplComp}
  \caption{Comparison of the fake and sideband distributions for charged hadron distribution obtained from the simulated \ttbar sample. The distributions are normalized to unity.}
  \label{fig_ttg_fit_faketemplcomp}
\end{figure}

\subsubsection{The Signal Template}
\label{sec_ttg_fit_sig}

In order to obtain a data driven signal template, a technique called random cone isolation is used \cite{Chatrchyan:2011qt}. In the end, the value of new isolation calculated with the random cone technique should not depend on the specific photon it was calculated from, but rather on the overall event shape. Effects like pile up, the underlying event or ECAL noise should impact the new isolation, thereby being incorporated into the data driven template fit. This reduces model dependencies and their impact on systematic uncertainties.  Starting from a reconstructed photon the following steps are used (see also Figure \ref{fig_ttg_fit_raco}):\\
\begin{itemize}
\item Starting with the reconstructed direction of a photon candidate a new axis (the "random-cone axis") is generated by isotropically rotating the photon axis in $\phi$ direction by a random angle between $0.8$ and $2 \pi - 0.8$. This ensures that an isolation cone around the new axis will not overlap with the isolation cone of the original photon.
\item The isolation of the new axis is checked. No particle flow jet with $\pt > \SI{20}{\giga \electronvolt}$ should be in a cone of $\Delta R < 0.8$. No photon with $\pt > \SI{10}{\giga \electronvolt}$ should be in a cone of $\Delta R < 0.8$ and no muon should be in a cone of $\Delta R < 0.4$. If any of these conditions is not fulfilled than a new random axis is determined until the criteria mentioned above are fulfilled. 
\item The charged hadron isolation of the new axis is calculated. For the calculation an area corresponding to the removed footprint of the original photon is removed from the calculation of the isolation of the new axis. This is done by rotating the supercluster of the original photon by the same angle the axis was rotated. Thereby the removed area is exactly he same as the original removed supercluster.
\end{itemize}

\begin{figure}[ht]
  \subfigure[]{
    \includegraphics[width = 0.4\textwidth] {Bilder/IsolationCone}
  }
    \subfigure[]{
    \includegraphics[width = 0.55\textwidth] {Bilder/RandomConeIso}
  }
  \caption{Sketch to explain the random cone technique. If the red circle represents the footprint of the photon candidate in picture a) and is then completely removed in picture b), the green circles representing the isolation should be the same for both picture a) and b). Consequently, the random cone isolation predicts the original isolation. \cite{RandCone_Talk,RandCone_AN}}
  \label{fig_ttg_fit_raco}
\end{figure}

The charged hadron isolation calculated with the random cone technique is called random cone isolation during the analysis. The random cone isolation in data is compared to the charged hadron isolation of the matched real photons in the simulated \ttgamma sample and with the random cone isolation in the \ttgamma sample in Figure \ref{fig_ttg_fit_raco}.\\
The random cone isolation shows minor differences between simulation and data, but there is a significant deviation in the first bin. \\
The templates of random cone isolation (from data) and charged hadron isolation of real photons (from simulation) are not compatible. The impact of the discrepancies will be discussed in Section \ref{sec_ttg_clo}. Whether they are due to mismodelling in the simulation, or whether the random cone isolation does not work as claimed cannot be determined. \enlargethispage{\baselineskip}

\begin{figure}[ht]
  \subfigure[Data: $I_{rand.co.}$, MC: $I_{ch.had.}$]{
    \includegraphics[width = 0.52\textwidth] {Bilder/RandCone_Comp_datamc_lin}
  }
    \subfigure[Data: $I_{rand.co.}$, MC: $I_{ch.had.}$]{
    \includegraphics[width = 0.52\textwidth] {Bilder/RandCone_Comp_DataMc_log}
 }
   \subfigure[Data: $I_{rand.co.}$, MC: $I_{rand.co.}$]{
    \includegraphics[width = 0.52\textwidth] {Bilder/RandCone_Comp_datamc2_lin}
  }
    \subfigure[Data: $I_{rand.co.}$, MC: $I_{rand.co.}$]{
    \includegraphics[width = 0.52\textwidth] {Bilder/RandCone_Comp_DataMc2_log}
 }
  \caption{Comparison of the charged hadron isolation calculated with the random-cone technique in data with the charged hadron isolation of real photons in the \ttgamma simulation in a) and b). In c) and d) the isolations computed via a random cone are compared between the \ttgamma simulation and data.}
  \label{fig_ttg_fit_raco}
\end{figure}

\FloatBarrier
\subsection{The Result of the Template Fit}
\label{sec_ttg_fit_res}

The template fit itself is performed with the "theta framework" \cite{theta} using a binned maximum likelihood fit. For the calculation of the uncertainty the Barlow-Beeston light method is used \cite{Barlow:1993dm,2011arXiv1103.0354C} expecting the uncertainty for each bin to be Gaussian. The two templates for fitting is shown in Figure \ref{fig_ttg_fit_tem}. They have sufficient separation power (in data).\\

\begin{figure}[ht]
\centering
    \includegraphics[width = 0.6\textwidth] {Bilder/TemplateFit_Templates}
  \caption{The templates used for the template fit normalized to integral. The two templates are described in detail in Section \ref{sec_ttg_fit_bkg}.}
  \label{fig_ttg_fit_tem}
\end{figure}

The result shows agreement between the fitted templates and data considering the uncertainties (see Figure \ref{fig_ttg_fit_res}). This is  reflected in the $\chi^2 / NDF = 1.2$. The result (shown in Equation \ref{eq_ttg_fit_res}) is compatible with the previous result \cite{CMS-PAS-TOP-13-011} while having a smaller statistical error: \\

\begin{equation}
N_{real} = 1894 \pm 76 ~~ \mathrm{and} ~~ N_{real,prev} = 1761 \pm 120 ~ (\mathrm{from} \cite{CMS-PAS-TOP-13-011}) .
\label{eq_ttg_fit_res}
\end{equation}

For the cross section calculation the number of real photons has to be corrected for indistinguishable background events using the purity of the selection.
The purity and the corrected result are shown in Equation \ref{eq_ttg_fit_rescor}:\\

\begin{equation}
\pi_{\ttgamma} = N^{sel}_{sig} / N^{sel} = \SI{66.7}{\percent} ~~ \mathrm{and} ~~ N^{sig}_{\ttgamma} = 1263 \pm 51 \;.
\label{eq_ttg_fit_rescor}
\end{equation}

The discrepancies between the templates used in the fit and the templates obtained from the matched real and fake photons from simulation (as described in Section \ref{sec_ttg_fit_bkg}) have to be investigated. Additionally, the fit procedure has to be validated. In order to do so, closure tests are used. The results are described in the following (see Section \ref{sec_ttg_clo}).

\begin{figure}[ht]
\centering

    \includegraphics[width = 0.67\textwidth] {Bilder/TemplateFit_Result}
  \caption{The result of the template fit. The two templates are scaled to the fit parameters $N_{real}$ and $N_{fake}$}
  \label{fig_ttg_fit_res}
\end{figure}

\FloatBarrier
\subsection{Assessment of the Template Fit: Closure Tests}
\label{sec_ttg_clo}

The purpose of the closure test is to validate a certain fit method by using pseudo experiments. Pseudo data is produced with a set of known parameters and the fit should be able to reproduce these parameters. Several different closure tests are used in this analysis. In order to assess the effect of the simulation of pile-up, different pile-up simulations are used. Specifically, samples with out-of-time (OOT) pile-up simulation and samples without OOT pile up simulation are employed. In detail, specifics used in this analysis are as follows (see also Figure \ref{fig_ttg_clo_dia} for an example):

\begin{itemize}
\item The pseudo data is obtained from different sets of samples. A list of the different closure tests and the details of the samples they use are found in Table \ref{tab_ttg_clo}. In general, all closure tests use a \ttgamma sample for the signal and a \ttbar sample for the background.
\item The pseudo data is generated by using two templates of matched real and fake photons. The template for the real photons is taken from the \ttgamma sample and the fake template is taken from the \ttbar sample. These two templates are then mixed with factors that are varied around the result from the template fit on data (see Section \ref{sec_ttg_fit_res}). For the real template, the factor rises from 1200 to 2900 in steps of 100. For the fake template, the factor rises from 2100 to 6100 in steps of 200. The steps and with them the respective template fits are continuously numbered from 1 to 17 (real) and 1 to 20 (fake). The stacked charged hadron isolation is considered as pseudo data. The errors on the pseudo data are statistical in respect to the final value, using Poisson errors for each bin.
\item The pseudo data from the last step is then fitted with a signal template taken from the \ttgamma sample as a template for real photons and the background template from the \ttbar sample as a template for fake photons. The uncertainties of the templates used for fitting are only statistical errors.
\item The template fit should then reproduce the factors used for the mixing of the pseudo data. 
\end{itemize}

\begin{table}[ht]
\caption{The different closure tests described in this section. The pseudo data column denotes the templates used for the mixing of pseudo data. The fitting column denotes the templates used for the template fit. The abbreviation OOT denotes samples with out-of-time pile-up.}
\centering
\begin{tabular}{|c|c|c|c|}

\hline
Closure Test & Samples & Pseudo Data & Fitting \\
\hline
\multirow{2}{*}{A} & \ttgamma & real & real \\
                   &  \ttbar  & fake & fake \\
\hline
\multirow{2}{*}{B} & \ttgamma + OOT & real & real \\
                   &  \ttbar  + OOT & fake & fake \\
\hline
\multirow{2}{*}{C} & \ttgamma & real & Random Cone \\
                   &  \ttbar  & fake & SBID \\
\hline
\multirow{2}{*}{D} & \ttgamma + OOT & real & Random Cone \\
                   &  \ttbar + OOT & fake & SBID \\
\hline
\multirow{2}{*}{E} & \ttgamma & real & Random Cone \\
                   &  \ttbar & fake & $\sigma_{i \eta i \eta}$ SB \\
\hline
\multirow{2}{*}{F} & \ttgamma + OOT & real & Random Cone \\
                   &  \ttbar + OOT & fake & $\sigma_{i \eta i \eta}$ SB \\
\hline
\multirow{2}{*}{G} & \ttgamma + OOT & real & Random Cone \\
                   &  \ttbar  & fake & SBID \\
\hline
\multirow{2}{*}{H} & \ttgamma & real & Random Cone \\
                   &  \ttbar  + OOT  & fake & SBID \\
\hline
                  
\end{tabular}
\label{tab_ttg_clo}
\end{table}

\begin{figure}[ht]
\centering
    \includegraphics[width = 0.55\textwidth] {Bilder/Closure_1}

  \caption{Flow chart for closure test B as an example.}
  \label{fig_ttg_clo_dia}
\end{figure}
\FloatBarrier
\subsubsection{Closure Tests A and B}

Closure tests A (without OOT pile-up) and B (with OOT pile-up) are sanity checks for the fit procedure. The templates used for the generation of the pseudo data are also used for the template fit. The result should exactly reproduce the input, as the templates are fitted to themselves (see Figure \ref{fig_ttg_clo_toy}).\\
The fit results exactly reproduce the input. This shows that the fit technically works on a trivial level. \\
The errors on the fit result for the number of fake photons are large, especially in closure test B containing OOT pile-up. These errors cannot be considered to be of a physical origin. The statistical errors of the templates are small, which might lead to a large uncertainty on the final result. Nevertheless, the errors are most likely caused by technical issues with the normalization of the templates. \enlargethispage{\baselineskip}\\

\begin{figure}[ht]
  \subfigure[without OOT pile-up]{
    \includegraphics[width = 0.52\textwidth] {Bilder/ClosureTest_toy_ex2}
  }
    \subfigure[with OOT pile-up]{
    \includegraphics[width = 0.52\textwidth] {Bilder/ClosureTest_toy_ex1}
 }
   \subfigure[without OOT pile-up]{
    \includegraphics[width = 0.52\textwidth] {Bilder/ClosureTest_toy_seq2}
    \label{fig_ttg_clo_toy_seq2}
  }
    \subfigure[with OOT pile-up]{
    \includegraphics[width = 0.52\textwidth] {Bilder/ClosureTest_toy_seq1}
    \label{fig_ttg_clo_toy_seq1}
 }
  \caption{The result of closure tests A and B. (a) and (b) show examples for the fit without (A) and with (B) OOT pile-up. (c) and (d) show the results for the complete closure test without (A) and with (B) OOT pile up.}
  \label{fig_ttg_clo_toy}
\end{figure}


\FloatBarrier
\subsubsection{Closure Test C}
\label{sec_ttg_clo_norm}

This closure test is performed using the \ttgamma and \ttbar samples without OOT pile-up (see Figure \ref{fig_ttg_clo_ex1} for an example result). There are significant differences between the pseudo data and the result of the fit. There is also the large error on the result for the number of fake photons, discussed in the last section. Still, the result of the template fit reproduces the factors used for the mixing.\\
\begin{figure}[ht]
\centering

    \includegraphics[width = 0.67\textwidth] {Bilder/ClosureTest_norm_ex}
    
  \caption{An example for the result for closure test C. The "mix" entry in the legend shows the factors used for the mixing of the pseudo data. The first number is the factor for fake photons, the second for real photons. }
  \label{fig_ttg_clo_ex1}
\end{figure}

The comprehensive results for the whole closure test are shown in Figure \ref{fig_ttg_clo_seqnorm}.\\
The large errors on the number of fake photons is again visible, but they are less pronounced in the second sequence of template fits where the number of fake photons is varied (see Figures \ref{fig_ttg_clo_seqnorm21} and \ref{fig_ttg_clo_seqnorm22}). This reinforces the conclusion that they are caused by technical issues.\\
The results of the whole closure test mostly reproduce the input parameters, at least within two times the statistical uncertainty. There is however a systematic structure in the results of the closure test. The reason might lie in an the overlap between signal and background templates. This will be further investigated with closure Tests E-H.  


\begin{figure}[ht]
  \subfigure[Fake photons]{
    \includegraphics[width = 0.52\textwidth] {Bilder/ClosureTest_norm_seq11}
  }
    \subfigure[Real photons]{
    \includegraphics[width = 0.52\textwidth] {Bilder/ClosureTest_norm_seq12}
 }
   \subfigure[Fake photons]{
    \includegraphics[width = 0.52\textwidth] {Bilder/ClosureTest_norm_seq21}
    \label{fig_ttg_clo_seqnorm21}
  }
    \subfigure[Real photons]{
    \includegraphics[width = 0.52\textwidth] {Bilder/ClosureTest_norm_seq22}
    \label{fig_ttg_clo_seqnorm22}
 }
  \caption{The results of closure test C for real and fake photons without OOT pile-up. In the sequence shown in a) and b) the number of real photons is varied, in sequence c) and d) the number of fake photons is varied.}
  \label{fig_ttg_clo_seqnorm}
\end{figure}

\FloatBarrier
\subsubsection{Closure Test D}
\label{sec_ttg_clo_oot}

Closure test D is performed using the \ttgamma and the \ttbar samples with OOT pile-up (see Figure \ref{fig_ttg_clo_ex2} for an example). In contrast to the closure test C, (see Section \ref{sec_ttg_clo_norm}) the fitted templates and the pseudo data agree with each other within their uncertainties. \\
The composition of the pseudo data is not reproduced by the fit results. While the number of fake photons is reproduced within the uncertainty, this is not meaningful, because of the uncertainties are large. The number of real photons seems to be significantly underestimated. \\

\begin{figure}[ht]
\centering

    \includegraphics[width = 0.67\textwidth] {Bilder/ClosureTest_oot_ex}
    
  \caption{An example for the result of closure Test D with OOT pile-up. The "mix" entry in the legend shows the factors used for the mixing of the pseudo data. The first number is the factor for fake photons, the second for real photons. }
  \label{fig_ttg_clo_ex2}
\end{figure}

This underestimation of the number of real photons becomes more apparent when looking at the whole closure test shown in Figure \ref{fig_ttg_clo_seqoot}. The number of real photons is consistently underestimated for both sequences. This will be discussed at the end of the section.\\
There is again a large error on the number of fake photons, especially in the sequence of fits where this number is kept constant. Since there are no other changes this hints at a technical reason. \\
The systematic structure visible in closure test C without OOT pile-up (see Figure \ref{fig_ttg_clo_seqnorm}) is less pronounced 
in this closure test with OOT pile-up. It should be due to a better separation between real and fake photons. Another reason could be the better agreement between the fitted templates and the pseudo data leading to a more stable fit result. This is further investigated with closure tests E-H.\\


\begin{figure}[ht]
  \subfigure[Fake photons]{
    \includegraphics[width = 0.52\textwidth] {Bilder/ClosureTest_oot_seq11}
  }
    \subfigure[Real photons]{
    \includegraphics[width = 0.52\textwidth] {Bilder/ClosureTest_oot_seq12}
 }
   \subfigure[Fake photons]{
    \includegraphics[width = 0.52\textwidth] {Bilder/ClosureTest_oot_seq21}
    \label{fig_ttg_clo_seqoot21}
  }
    \subfigure[Real photons]{
    \includegraphics[width = 0.52\textwidth] {Bilder/ClosureTest_oot_seq22}
    \label{fig_ttg_clo_seqoot22}
 }
  \caption{The results of closure test D for real and fake photons with OOT pile-up. In the sequence shown in a) and b) the number of real photons is varied, in sequence c) and d) the number of fake photons is varied.}
  \label{fig_ttg_clo_seqoot}
\end{figure}
\FloatBarrier

\subsubsection{Closure Tests E-H}
\label{sec_ttg_clo_test}

In order to further investigate the validity of the fit more tests are performed.\\
In closure tests E and F the template for fake photons used for the fit is changed to the alternative template obtained from the sideband of only $\sigma_{i \eta i \eta}$ (see Section \ref{sec_ttg_fit_sig}). There is little change in the results of the template fit (see Figure \ref{fig_ttg_clo_chte}), which shows that the fit is stable. The systematic trend in the closure test (F) with OOT pile-up is still significant (see Figure \ref{fig_ttg_clo_chte1}). This shows that the separation between the templates for real and fake photons is likely not the cause for this effect, as the alternative fake template should provide a different degree of separation.\\

\begin{figure}[ht]
   \subfigure[without OOT pile-up]{
    \includegraphics[width = 0.52\textwidth] {Bilder/ClosureTest_norm_ChTe}
    \label{fig_ttg_clo_chte1}
  }
    \subfigure[with OOT pile-up]{
    \includegraphics[width = 0.52\textwidth] {Bilder/ClosureTest_norm_ChTe2}
 }
  \caption{Results of the closure tests with an alternative template used to fit the amount of fake photons. (a) and (b) show closure tests E and F.}
  \label{fig_ttg_clo_chte}
\end{figure}

For closure tests G and H, the templates used for the fit and the pseudo data generation were interchanged: The \ttgamma sample with OOT pile-up was used with the \ttbar sample without OOT pile-up (see Figures \ref{fig_ttg_clo_test_11} and \ref{fig_ttg_clo_test_12}) in closure test G. In closure test H, the \ttgamma sample with OOT pile-up was used with the \ttbar sample with OOT pile-up (see Figures \ref{fig_ttg_clo_test_21} and \ref{fig_ttg_clo_test_22}). The results show that in closure test G the number of real photons is overestimated instead of underestimated compared to closure test D. In closure test H the underestimation of the number of real photons is more significant than in closure test D. This shows that the differences from the expectation shown in closure test D cannot be pinned down to a single sample, but rather that it is the combination of the samples that lead to the discrepancies.

\begin{figure}[ht]
  \subfigure[\ttgamma + OOT with \ttbar]{
    \includegraphics[width = 0.52\textwidth] {Bilder/Closure_Test_oot_testmc}
    \label{fig_ttg_clo_test_11}
  }
    \subfigure[\ttgamma + OOT with \ttbar]{
    \includegraphics[width = 0.52\textwidth] {Bilder/Closure_Test_oot_testmc2}
    \label{fig_ttg_clo_test_12}
 }
   \subfigure[\ttgamma with \ttbar + OOT]{
    \includegraphics[width = 0.52\textwidth] {Bilder/Closure_Test_oot_test2mc}
    \label{fig_ttg_clo_test_21}
  }
    \subfigure[\ttgamma with \ttbar + OOT]{
    \includegraphics[width = 0.52\textwidth] {Bilder/Closure_Test_oot_test2mc2}
    \label{fig_ttg_clo_test_22}
 }
  \caption{The results of the closure tests G and H with interchanged templates.}
  \label{fig_ttg_clo_test}
\end{figure}

\FloatBarrier
\subsubsection{Conclusion}

The tests that are performed to understand the unsatisfying result of closure test D provide no conclusive result. One reason could be the mismodeling of the charged hadron isolation in simulation. This is motivated by the comparison of the charged hadron isolation in simulation and data (see Figure \ref{fig_ttg_fit_chhadnm1}) where the photons are shown to be less isolated in simulation, even without OOT pile-up. This mismodelling is a good reason to apply a data driven template fit.\\
Another assumption would be that the random cone isolation technique does not work in this case, as evidenced by the disagreement between the template obtained with random cone isolation and the template of matched real photons obtained from simulation (see Figure \ref{fig_ttg_fit_raco}). However, this disagreement can also be explained by mismodelling in simulation. \\
The large errors on the number of background photons are found in all closure tests. In the sequence where the number of background photons is varied they are in the same number of magnitude as in the template fit performed completely data driven (see Figure \ref{fig_ttg_fit_res}). Consequently, they can be explained with the different statistic uncertainties in the templates, due to the high number of simulated events. In the sequence where the number of background events is kept constant, the errors can no longer be explained 
by a different number of events in the templates. These errors are most likely due to technical issues.\\ 
The number of background events does not directly influence the measurement. Additionally, the number of signal photons is stable in the closure test irrespective of the uncertainty on the number of background photons. Therefore, these uncertainties do not affect the measurement and do not have to be considered for the final result. \\
Systematic trends found in the closure tests are most significant at the edges of the values for signal and background photons. Consequently, they should have less impact on the value found in data, which lies near the center of the values checked with the closure tests. The trend is also less pronounced in the closure tests including OOT pile-up, so it might be caused by the pile-up simulation. \\ 
Since the simulation of pile-up effects remains difficult and the template fit works works very well in data (see Figure \ref{fig_ttg_fit_res}), the conclusion for this analysis is that the result of the data driven template fit is valid.\\
In order to estimate the effect of different pile-up scenarios, the underestimation of signal photons found in closure test D (see Figure \ref{fig_ttg_clo_seqoot}) is propagated to the final result as a systematic uncertainty. This uncertainty also covers the systematic trends in the closure test, as they are less significant. The effects found in the closure tests should still be investigated further and in the outlook (see Chapter \ref{sec_out}) some suggestions are made.

\newpage
\section{Systematic Uncertainties}
\label{sec_ttg_sys}

The systematic uncertainties are calculated according to \cite{CMS-PAS-TOP-13-011}. The specifics for each source are given below and the results are summarized. Generally these uncertainties are obtained by changing the corresponding parameter, then running the whole analysis chain again and interpreting the deviation of the result as systematic uncertainty. The results are found in Table \ref{tab_ttg_sys}.

\begin{itemize}
\item \textbf{Pile-up}: The uncertainty on the pile-up distribution is mainly due to the uncertainty on the overall luminosity and the inelastic proton-proton cross section. This analysis assumes approximately 20 additional proton-proton interactions per event, based on a proton-proton cross section of \SI{69.4}{\milli \barn} \cite{Antchev:2011vs}. The systematic uncertainty is calculated according to the uncertainty of the cross section of \SI{5.9}{\percent}.  
\item \textbf{Out-of-time pile-up}: The uncertainty due to OOT pile-up is estimated by mixing pseudo data from samples with OOT pile-up and then fitting it with the random cone and SBID templates used in the template fit. The difference between the fit results and the factors applied in the mixing is used as systematic uncertainty.
\item \textbf{Signal normalization}: This uncertainty is calculated by changing the signal yield by \SI{\pm 25}{\percent}. This includes the variation of the kinematic scales for the cross section calculation \cite{Melnikov:2011ta}.
\item \textbf{Background normalization}: The background normalization is varied within the individual cross section uncertainties \cite{CMS-PAS-TOP-13-011}. The single deviations are then summed up quadratically.
\item \textbf{Hadronization/Showering}: The impact of the model used for the simulation is estimated by comparing samples simulated with POWHEG \cite{Alioli:2011as,Nason:2004rx,Frixione:2007vw,Alioli:2010xd} and subsequently showered and hadronized with either PYTHIA \cite{Sjostrand:2006za} or HERWIG \cite{Corcella:2000bw}.
\item \textbf{Generator}: This uncertainty is evaluated by comparing POWHEG and MADGRAPH samples showered with PYTHIA and POWHEG and MC@NLO \cite{Frixione:2002ik,Frixione:2003ei} samples showered with HERWIG. The largest deviation is taken as systematic uncertainty.
\item \textbf{PDF uncertainties}: Due to technical reasons they are not calculated for this analysis. In the previous analysis their impact was negligible \cite{CMS-PAS-TOP-13-011}.
\item \textbf{Top quark \pt}: There are differences between the \pt distribution of top quarks in data and simulation \cite{CMS-PAS-TOP-12-027}. Therefore simulated samples are reweighted according to the data distribution. In order to evaluate the uncertainty caused by the reweighing, the weights are either doubled or set to one.
\item \textbf{Identification of b-jets}: The impact of b-tagging has to be corrected in simulation depending on the \pt , $\eta$ and flavor of a jet \cite{CMS-PAS-BTV-13-001}. The b-tagging uncertainty is estimated by varying these scaling factors within their uncertainties.
\item \textbf{Jet energy correction}: The jet energy correction factors are varied within their uncertainty.
\item \textbf{Jet energy resolution}: The jet energy resolution is smeared in simulation to match the resolution found in data. The smearing factors are varied within their uncertainties.
\end{itemize}

The dominant uncertainties are due to out-of-time pile-up, signal cross section and the generator that is chosen.

\begin{table}[ht]
\centering
\caption{Uncertainties on the relation between \ttbar and \ttgamma cross sections $R^{vis}$ and the \ttgamma cross section $\sigma_{\ttgamma}$.}
\label{tab_ttg_sys}
\begin{tabular}{l c c} \\

\hline

\hline

Source & \multicolumn{2}{c}{Uncertainty (\%)} \\

& $R^{vis}$ & $\sigma_{\ttgamma}$ \\

\hline

Statistical & 4.1 & 4.1 \\

\hline

Systematic & 18.2 & 19.4 \\

\hline

Individual contributions: & & \\

\;\;\;pileup & 0.4 & 0.3 \\

\;\;\;out-of-time pileup & 8.2 & 8.2 \\

\;\;\;$\sigma_{\mathrm{ sig}}$ & 9.0 & 9.0 \\

\;\;\;$\sigma_{\mathrm {bkg}}$ & 1.4 & 1.4 \\

\;\;\;shower./hadr. & 3.0 & 3.0 \\

\;\;\;generator & 13.0 & 13.0 \\

\;\;\;top-quark \pt & 0.5 & 0.3 \\

\;\;\;b-tag & 0.3 & 0.3 \\

\;\;\;JEC & 1.9 & 1.7 \\

\;\;\;JER & 0.3 & 0.4 \\

\hline

\textbf{Total} & 18.7 & 19.8 \\

\hline

\hline

\end{tabular}
\end{table}

\section{The \ttgamma Cross Section}
\label{sec_ttg_crossec}

The ratio between the \ttbar and the \ttgamma cross section is determined according to Equation \ref{eq_ttg_cros_R}:

\begin{equation}
R \equiv \frac{R^{vis}}{\epsilon^{vis}_{\gamma}} = \frac{\sigma_{\ttgamma}}{1} \cdot \frac{1}{\sigma_{\mathrm{t}\overline{\mathrm{t}}}} = \frac{N^{sig}_{\ttgamma}}{\epsilon^{vis}_{\gamma} \epsilon_{\gamma}} \cdot \frac{1}{N^{presel} \pi_{\mathrm{t}\overline{\mathrm{t}}}}.
\label{eq_ttg_cros_R}
\end{equation}

Here the purity of the \ttbar selection (see Equation \ref{eq_ttg_cros_purtt}) and the efficiency of the photon selection (see Equation \ref{eq_ttg_cros_effgam}) are determined from simulation:

\begin{equation}
\pi_{\ttbar} = N_{\ttbar}^{presel} / N^{presel} = \SI{84.3}{\percent} ~~\mathrm{and}
\label{eq_ttg_cros_purtt}
\end{equation}

\begin{equation}
\epsilon_{\gamma}^{vis} \epsilon_{\gamma} = N_{\ttgamma}^{sel} / N_{\ttgamma}^{presel} = \SI{62.9}{\percent}.
 \label{eq_ttg_cros_effgam}
\end{equation}

The luminosity, detector acceptance and the preselection efficiency cancel out. The result is shown in Equation \ref{eq_ttg_cros_Rres}:

\begin{equation}
R^{vis} = (0.93 \;\pm 0.04\,\mathrm{stat}\;\pm 0.17\,\mathrm{syst})\cdot 10^{-2} ~~\mathrm{and}~~ R = (1.15 \;\pm 0.05\,\mathrm{stat}\;\pm 0.21\,\mathrm{syst})\cdot 10^{-2}.
\label{eq_ttg_cros_Rres}
\end{equation}

The cross section of the \ttgamma process is calculated by multiplying $R$ with $\sigma_{\ttbar}$, taken from the official CMS result  \cite{CMS-PAS-TOP-12-007} (see Equation \ref{eq_ttg_cros_restt}):

\begin{equation}
\sigma_{\ttbar}^{\mathrm{CMS}} = \SI{227 \pm 15}{\pico \barn}\;.
\label{eq_ttg_cros_restt}
\end{equation}

This leads to a the result for the \ttgamma cross section shown in Equation \ref{eq_ttg_cros_res}:

\begin{equation}
\sigma_{\ttgamma} \;=\; R\;\cdot \;\sigma_{\ttbar} \; =\; 2.6\;\pm \;0.1\mathrm{ (stat.)} \;\pm \;0.5\mathrm{ (syst.)}\; \SI{}{\pico \barn}\;.
\label{eq_ttg_cros_res}
\end{equation}

This is consistent with both the theoretical prediction of $\sigma_{\ttgamma} = \SI{1.8 \pm 0.5}{\pico \barn}$ \cite{Melnikov:2011ta} and the official CMS result of $\sigma_{\ttgamma}^{CMS} = \SI{2.4\pm 0.6}{\pico \barn}$ \cite{CMS-PAS-TOP-13-011}.