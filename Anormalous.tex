\chapter{Anormalous Couplings}

In this chapter a benchmark study for anomalous couplings of the top quark is presented. A generator study is used to evaluate the sensitivity of a possible \ttgamma analysis for a non zero magnetic dipole moment of the top quark.
\section{Theoretical Background}
\subsection{The top-photon Vertex}

As previously mentioned \todo{Link} the \ttgamma Lagrangian can be described as shown in  Equation \ref{eq_ano_ttg} \todo{ Zitat}.

\begin{equation}
  \mathcal{L}_{\ttgamma} = -e Q_t \overline{t} \gamma^{\mu} t A_{\mu} - e \overline{t} \frac{i \sigma^{\mu \nu}}{m_t} (d_V^{\gamma} + i d_a^{\gamma} \gamma_5) t A_{\mu}
  \label{eq_ano_ttg}
  \end{equation}

Here the first term is the contribution from the standard model, where the Lagrangian is proportional to the top-quark charge $Q_t$.\\
In the secind term $d_a^{\gamma}$ and $d_V^{\gamma}$ are a possible electric- or  magnetic-dipole moment of the top quark, which would lead to loop corrections to the coupling strenght of photons to top quarks. The electric-dipole moment is supposed to be heavily suppressed in the Standard Model \todo{find Citation} and the magnetic-dipole moment is predicted to be zero. Consequently, any meaningfull contribution should come from physics beyond the standard model. The magnetic-dipole moment was already discussed in another analysis \todo{Cite Markus}, therefore this analysis will look into a possible magnetic-dipole moment.\\
The  magnetic dipole moment can be connected to a hypothetical non-point like charge distribution, which would lead to a non zero radius of the top quark. Following \todo{Citation} the relation is shown in Equation \ref{eq_ano_rad}.

\begin{equation}
R_t = \frac{\sqrt{6}}{\Lambda_{\ast}} \hspace{1.cm} d_V^{\gamma} = \frac{\rho m_t^2}{\Lambda_{\ast}}
\label{eq_ano_rad}
\end{equation}

Here $\rho$ is a $\mathcal{O}(1)$ number, $\Lambda_{\ast}$ is the scale factor and $m_t$ is the top-quark mass.\\
A non zero magnetic-dipole moment could have an effect on \ttgamma events. In this chapter several observables are investigated in order to assess their sensitivity. The production cross section is also examined.

\section{Investigation using Simulated Events}

The generator WHIZARD is used to generate events for different top-quark magnetic dipole moments. The factorized $2 \to 7$ approach explained in Section \ref{ch_simu_con} and shown in Equation \ref{eq_ano_fac} is used.

\begin{equation}
\begin{split}
& \mathrm{pp}\to \ttgamma,  \mathrm{t} \to \mathrm{bxx},  \overline{\mathrm{t}} \to \overline{\mathrm{b}} \mathrm{xx} \\
& \mathrm{pp}\to \mathrm{t} \overline{\mathrm{t}},  \mathrm{t} \to \mathrm{bxx} \gamma,  \overline{\mathrm{t}} \to \overline{\mathrm{b}} \mathrm{xx} \\
& \mathrm{pp}\to \mathrm{t} \overline{\mathrm{t}},  \mathrm{t} \to \mathrm{bxx},  \overline{\mathrm{t}} \to \overline{\mathrm{b}} \mathrm{xx} \gamma \\
&  x\ = \ \mathrm{u}, \mathrm{d}, \mathrm{s}, \mathrm{c}, \overline{\mathrm{u}}, \overline{\mathrm{d}},  \overline{\mathrm{s}},  \overline{\mathrm{c}}, \mathrm{e}^{\pm}, \mu^{\pm}, \tau^{\pm}, \nu_{\mathrm{e},\mu,\tau}, \overline{\nu}_{\mathrm{e},\mu,\tau} 
\end{split}
\label{eq_ano_fac}
\end{equation}

These events also include photons which are not radiated from the top quark making them less sensitive for anomalous couplings, but it is difficult to separate these processes.\\
Only the output of the matrix element generator is compared here. The detector response or any other uncertainties are not included here.
There should be a clear difference between the different scenarios, otherwise a real analysis using measured data is not reasonable.\\
The values chosen here for the multiple scenarios are shown in equation \ref{eq_ano_val}.

\begin{equation}
d_V^{\gamma} = 0.,\; 0.06,\; 0.12,\; 0.18,\; 0.30
\label{eq_ano_val}
\end{equation}

The highest value of $d_V^{\gamma} = 0.30$ corresponds to a "radius" of $R_{\mathrm{t}} = \SI{0.65d-3}{\femto \meter}$. The numbers were basically chosen arbitrarily. One point of orientation for the choice where the limits on the magnetic dipole moment of the top quark \todo{Citation}, the other was the radius of the proton $R_{\mathrm{p}} \approx \SI{1}{\femto \meter}$. The proton radius should be significantly greater than the radius of the top quark. \\
As a first point for investigation the calculated cross sections for the $\mathrm{pp} \to \ttgamma $ process can be compared (see Table \ref{tab_ano_crosssec}). No errors are given here, because of the technical complications when calculating these cross sections from the WHIZARD output (described in Section \todo{Link}). An error caclulation is possible and would have to be included in an analysis, but for this benchmark study it is considered unneccesary.

\begin{table}[ht]
\centering
    \caption{Cross sections for the $\mathrm{pp} \to \ttgamma$ process as calculated by WHIZARD. Multiple scenarios for a possible anormalous magnetic dipole moment are compared. No errors are given.}
    \begin{tabular}{| l | l |}

    \hline
    $d_{V}^{\gamma}$ & $\sigma{\ttgamma} [\SI{}{\femto \barn}]$ \\
    \hline
    $0$ &  $1408$\\
    \hline
    $0.06$ & $1408 $ \\
    \hline
    $0.12$ & $1407 $ \\
    \hline
    $0.18$ & $1420 $ \\
    \hline
    $0.30$ &  $1443 $\\
    \hline
    \end{tabular}
     \label{tab_ano_crosssec}
\end{table}

The values for the cross section show little separation power, the first three being clearly within the uncertainty of the calculation. The difference in cross section between Standard Model ($d_V^{\gamma} = 0$)  and a scenario with $d_V^{\gamma} = 0.3$ is $\Delta \sigma_{\ttgamma} = \SI{35}{\femto \barn}$. Such a difference can as of now not be measured with the CMS detector  (\todo{Maybe quote Heiners uncertainty}).\\
In the next step several kinematic observables are compared. They might offer more separation power than the comparison of the cross section.

\begin{figure}
  \subfigure[$\pt(\gamma)$]{
    \includegraphics[width = 0.67\textwidth] {Bilder/photonEt_Anomalous}
  }
  \\
    \subfigure[$\Delta \mathrm{R} (\gamma, \mathrm{q})$]{
    \includegraphics[width = 0.67\textwidth] {Bilder/dRphotonQ_Anomalous}
  }
      \subfigure[$\Delta \mathrm{R} (\gamma, \mathrm{t})$ \todo{Caption}]{
    \includegraphics[width = 0.67\textwidth] {Bilder/dRphotonT_Anomalous}
  }
  \caption{Comparison between the MADGRAPH and WHIZARD generators. All distributions are normalized to unity. \todo{Caption}}
  \label{fig_ano_comp_1}
\end{figure}

The results are shown in Figure \ref{fig_ano_comp_1}. These comparisons show that the variables investigated here show no separation power.  The difference between the scenarios should be significant, in order to have even a chance at separating the two scenarios after the full detector simulation.\\
A few more complicated observables are investigated in Figure \ref{fig_ano_comp_2}. In Figure \ref{fig_ano_comp_2_1} the transversal momentum of the top quark is shown in two bins. A similar observable is used in another study on anomalous top couplings \todo{cite Markus}. In Figure \ref{fig_ano_comp_2_2} the angle between the two top quarks is drawn. This observable is insired by spiin-correlation measurements \todo{Cite Berni}. In a real analysis they would have to be reconstructed, which is difficult or the observable would have to be changed to a angular correlation of the decay products.\\
Even with these observables the separation power does not increase significantly. For the phase space tested here, the data we have in the moment is insufficient for a measurement.   

\begin{figure}
  \subfigure[$\pt(\gamma)$]{
    \includegraphics[width = 0.67\textwidth] {Bilder/Rebin_Anomalous}
    \label{fig_ano_comp_2_1}
  }
  \\
    \subfigure[$\theta (\mathrm{t},\overline{\mathrm{t}})$]{
    \includegraphics[width = 0.67\textwidth] {Bilder/topangle_Anomalous}
    \label{fig_ano_comp_2_2}
  }
  \caption{Comparison between the MADGRAPH and WHIZARD generators. All distributions are normalized to unity. \todo{Caption}}
  \label{fig_ano_comp_2}
\end{figure}

To assess the impact of a very strong anomalous magnetic coupling a few extreme scenarios were generated. The values for the electromagnetic coupling are shown in Equation \ref{eq_ano_val_ex}. They are unrealistic as such a high anomalous coupling would have probably been detected already.

\begin{equation}
d_V^{\gamma} = 0.,\; 0.30,\; 1.0,\;1.5,\; 5.0
\label{eq_ano_val_ex}
\end{equation}

Again, the cross sections are investigated (see Table \ref{tab_ano_crosssec_ex}). The different scenarios are now clearly distinct and it would be possible to set a limit on the coupling using the CMS result of \todo{Result or Link here}. Taking into account the large corrections factors between leading order and next to leading order discussed in Chapter \todo{Link} setting limits is not feasible. The technical difficulties when calculating the cross section from the WHIZARD output also make these results less suited for a limit calculation. \\

\begin{table}[ht]
\centering
    \caption{Cross sections for the $\mathrm{pp} \to \ttgamma$ process as calculated by WHIZARD. Multiple scenarios for a possible a very high anormalous magnetic dipole moment are compared. No errors are given. These are used to assess the general impact of a high anomalous coupling.}
    \begin{tabular}{| l | l |}

    \hline
    $d_{V}^{\gamma}$ & $\sigma{\ttgamma} [\SI{}{\femto \barn}]$ \\
    \hline
    $0$ &  $1408$\\
    \hline
    $0.3$ & $1443 $ \\
    \hline
    $1.0$ & $1723 $ \\
    \hline
    $1.5$ & $2161 $ \\
    \hline
    $5.0$ &  $12874 $\\
    \hline
    \end{tabular}
     \label{tab_ano_crosssec_ex}
\end{table}

Different kinematic observables are also studied (see Figure \ref{fig_ano_comp_ex}). The different scenarios are separable, especially the one with $d_{V}^{\gamma} = 5.0$ shows distinctive behaviour. A limit calculation would be possible using either more different simulated scenarios and interpolation or  matrix element reweighting to achieve a continouus distribution. These methods would require a lot of effort and are not further pursued in this analysis.

\begin{figure}
  \subfigure[$\pt(\gamma)$]{
    \includegraphics[width = 0.67\textwidth] {Bilder/photonEt_Anomalous_extreme}
  }
  \\
    \subfigure[$\Delta \mathrm{R} (\gamma, \mathrm{q})$]{
    \includegraphics[width = 0.67\textwidth] {Bilder/dRphotonQ_Anomalous_extreme}
  }
      \subfigure[$\Delta \mathrm{R} (\gamma, \mathrm{t})$ \todo{Caption}]{
    \includegraphics[width = 0.67\textwidth] {Bilder/dRphotonT_Anomalous_extreme}
  }
  \caption{Comparison between the MADGRAPH and WHIZARD generators. All distributions are normalized to unity. \todo{Caption}}
  \label{fig_ano_comp_ex}
\end{figure}

\section{Conclusion}

The benchmark study shows that the $\mathrm{pp} \to \ttgamma$ channel is in the moment not suited well for a high precision measurement of a possible magnetic moment of the top quark. While the investigation of a hypothetical very high coupling shows that a limit could be set using either the event kinematics or a more precisely calculated cross section, this would require lot of time and effort. The result would likely be unsatisfactory, so a further investigation seems not sensible in the moment. In all likelihood it will remain challenging even after the next LHC run has delivered its results. 