\chapter{Anormalous Couplings}

In this chapter a benchmark study for anomalous couplings of the top quark is presented. A generator study is used to evaluate the sensitivity of a possible \ttgamma analysis for a non zero magnetic dipole moment of the top quark.
\section{Theoretical Background}
\subsection{The top-photon Vertex}

As previously mentioned \todo{Link} the \ttgamma Lagrangian can be described as shown in  Equation \ref{eq_ano_ttg} \todo{ Zitat}.

\begin{equation}
  \mathcal{L}_{\ttgamma} = -e Q_t \overline{t} \gamma^{\mu} t A_{\mu} - e \overline{t} \frac{i \sigma^{\mu \nu}}{m_t} (d_V^{\gamma} + i d_a^{\gamma} \gamma_5) t A_{\mu}
  \label{eq_ano_ttg}
  \end{equation}

Here the first term is the contribution from the standard model, where the Lagrangian is proportional to the top-quark charge $Q_t$.\\
In the secind term $d_a^{\gamma}$ and $d_V^{\gamma}$ are a possible electric- or  magnetic-dipole moment of the top quark, which would lead to loop corrections to the coupling strenght of photons to top quarks. The electric-dipole moment is supposed to be heavily suppressed in the Standard Model \todo{find Citation} and the magnetic-dipole moment is predicted to be zero. Consequently, any meaningfull contribution should come from physics beyond the standard model. The magnetic-dipole moment was already discussed in another analysis \todo{Cite Markus}, therefore this analysis will look into a possible magnetic-dipole moment.\\
The  magnetic dipole moment can be connected to a hypothetical non-point like charge distribution, which would lead to a non zero radius of the top quark. Following \todo{Citation} the relation is shown in Equation \ref{eq_ano_rad}.

\begin{equation}
R_t = \frac{\sqrt{6}}{\Lambda_{\ast}} \hspace{1.cm} d_V^{\gamma} = \frac{\rho m_t^2}{\Lambda_{\ast}}
\label{eq_ano_rad}
\end{equation}

Here $\rho$ is a $\mathcal{O}(1)$ number, $\Lambda_{\ast}$ is the scale factor and $m_t$ is the top-quark mass.\\
A non zero magnetic-dipole moment could have an effect on \ttgamma events. In this chapter several observables are investigated in order to assess their sensitivity. The production cross section is also examined.

\section{Investigation with Simulated Events}

The generator WHIZARD is used to generate events for different top-quark magnetic dipole moments. The factorized $2 \to 7$ approach explained in Section \ref{ch_simu_con} and shown in Equation \ref{eq_ano_fac} is used.

\begin{equation}
\begin{split}
& \mathrm{pp}\to \ttgamma,  \mathrm{t} \to \mathrm{bxx},  \overline{\mathrm{t}} \to \overline{\mathrm{b}} \mathrm{xx} \\
& \mathrm{pp}\to \mathrm{t} \overline{\mathrm{t}},  \mathrm{t} \to \mathrm{bxx} \gamma,  \overline{\mathrm{t}} \to \overline{\mathrm{b}} \mathrm{xx} \\
& \mathrm{pp}\to \mathrm{t} \overline{\mathrm{t}},  \mathrm{t} \to \mathrm{bxx},  \overline{\mathrm{t}} \to \overline{\mathrm{b}} \mathrm{xx} \gamma \\
&  x\ = \ \mathrm{u}, \mathrm{d}, \mathrm{s}, \mathrm{c}, \overline{\mathrm{u}}, \overline{\mathrm{d}},  \overline{\mathrm{s}},  \overline{\mathrm{c}}, \mathrm{e}^{\pm}, \mu^{\pm}, \tau^{\pm}, \nu_{\mathrm{e},\mu,\tau}, \overline{\nu}_{\mathrm{e},\mu,\tau} 
\end{split}
\label{eq_ano_fac}
\end{equation}

These events also include photons which are not radiated from the top quark making them less sensitive for anomalous couplings, but it is difficult to separate these processes.\\
Only the output of the matrix element generator is compared here. The detector response or any other uncertainties are not included here.
There should be a clear difference between the different scenarios, otherwise a real analysis using measured data is not reasonable.\\
The values chosen here for the multiple scenarios are shown in equation \ref{eq_ano_val}.

\begin{equation}
d_V^{\gamma} = 0.,\; 0.06,\; 0.08,\; 0.12,\; 0.18,\; 0.30
\label{eq_ano_val}
\end{equation}

The highest value of $d_V^{\gamma} = 0.30$ corresponds to a "radius" of $R_{\mathrm{t}} = \SI{0.65d-3}{\femto \meter}$. The numbers were basically chosen arbitrarily. One point of orientation for the choice where the limits on the magnetic dipole moment of the top quark \todo{Citation}, the other was the radius of the proton $R_{\mathrm{p}} \approx \SI{1}{\femto \meter}$. The proton radius should be significantly greater than the radius of the top quark. \\


\begin{figure}
  \subfigure[$\pt(\gamma)$]{
    \includegraphics[width = 0.67\textwidth] {Bilder/photonEt_Anomalous}
  }
  \\
    \subfigure[$\Delta \mathrm{R} (\gamma, \mathrm{q})$]{
    \includegraphics[width = 0.67\textwidth] {Bilder/dRphotonQ_Anomalous}
  }
      \subfigure[$\Delta \mathrm{R} (\gamma, \mathrm{t})$ \todo{Caption}]{
    \includegraphics[width = 0.67\textwidth] {Bilder/dRphotonT_Anomalous}
  }
  \caption{Comparison between the MADGRAPH and WHIZARD generators. All distributions are normalized to unity.}
  \label{fig_ano_comp}
\end{figure}

The results are shown in Figure \ref{fig_ano_comp}. 