\chapter{Outlook}
\label{sec_out}

The different parts of this analysis could be improved in multiple ways. Some ideas are given in this section. \\
The new simulation strategy for \ttgamma events (see Chapter \ref{sec_simu}) could be further developed by implementing it in different generators. Especially, aMC@NLO is an interesting candidate \cite{Alwall:2014hca}, as it is a matrix element generator working on NLO level. Eventual differences between aMC@NLO and MADGRAPH or WHIZARD would provide for a better understanding of the simulation and the uncertainties associated with it. \\ 
 \\
The measurement of an anomalous magnetic moment of the top quark (see Chapter \ref{sec_ano}) in the \ttgamma sector is challenging. The resulting sensitivity for the top radius is not competitive with chromomagnetic measurements. As such, an improvement of the benchmark study would be possible, but the measurement will remain challenging in the future. \\
\\
The \ttgamma cross section measurement is dominated by systematic uncertainties (see Chapter \ref{sec_ttg}), so this should be the starting point for any improvement. One of the three dominant systematic uncertainties is the uncertainty of the theoretical prediction of the \ttgamma cross section. This uncertainty can be lowered by a better NLO cross section calculation, for example the cross section calculation for the new simulation strategy (see Section \ref{sec_simu_conc}). By implementing the \ttgamma simulation based on this strategy the uncertainty on the final result caused by the error on the signal cross section should be significantly reduced.\\
The second of the dominant systematic uncertainties is caused by the generator chosen for the simulation of \ttbar background events. Since this error is mostly caused by differences in the purity of the preselection of \ttbar events, they can be addressed by a data driven calculation of this purity. A discriminating variable could be used for a template fit to extract the normalization, which would likely be less dependent on the generator. One possibility for a variable is the combined mass of the three leading jets, often called $M3$. This variable should be able to distinguish between \ttbar events and \ttbar like background events.\\
The last of the dominant uncertainties is caused by out-of-time pile-up and will likely be the hardest to reduce. One possibility is to find a discriminating variable that is less affected by the pile-up simulation or that is generally better modeled in simulation. The photon isolation could be an option that might also provide more separation against neutral pions. Another option could be the use of multiple variables for a combined template fit to increase the separation power. The way the data driven templates are obtained should also be scrutinized. Especially, the sideband for the template of non-signal like photons might be changed to increase the separation power or reduce the dependence on pile-up. \\
\\
As a final conclusion the investigation of \ttgamma events should remain an important research topic in the future. When the LHC reaches higher energy a well understood top-photon vertex will be of special importance to a lot of searches for new physics. Even though it might be challenging, the efforts will likely be a relevant part of top-quark physics in the next years.\enlargethispage{\baselineskip} \\