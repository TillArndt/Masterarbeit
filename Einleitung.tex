\chapter{Introduction}

If there is one sentence that defines particle physics it certainly is the question: What are the ingredients of matter ? \\
There have been numerous answers over the centuries. Today our best answer is summed up in the Standard Model (SM) of particle physics. The SM has been wildly successful over the last years in predicting the experimental results. The latest success is certainly the discovery of the Higgs boson by the CMS and ATLAS experiments in 2012 \cite{Aad:2012tfa} \cite{Chatrchyan:2012ufa} after it had been first predicted in 1964 \cite{Englert:1964et} \cite{Higgs:1964ia}. This discovery at last offers the reason to why elementary particles have mass. \\
Through all the success of the SM several fundamental questions remain unanswered: Can the three forces in the SM be unified ? What is the solution to the Higgs hierarchy problem ? What is dark matter ? Can gravity be described within a quantum field theory ? \\
These questions can probably not be answered within the framework of the SM. Through the years there have been a lot of proposed extensions and alternatives to the SM, yet no conclusive proof has been found until today that one of them is correct. Here the Large Hadron Collider (LHC) at Cern, Geneva, comes into play. It is one of the instruments allowing for the probing of the SM at previously unknown energy scales. \\
Top quarks belong to the particles commonly produced at the LHC. They are also a good object to test theories beyond the SM (BSM). Many BSM theories predict stronger differences to the SM for heavy or third generation leptons \cite{AguilarSaavedra:2008zc}. Additionally, top quark couplings can be measured without interference from hadronization. These reasons make the top sector an ideal target for generic searches for BSM physics. \\
This analysis focuses on the top-photon vertex. It uses data with an integrated luminosity of $\mathcal{L}_{int} = \SI{19.7}{\per \femto \barn}$ taken at the LHC with the CMS detector at a center of mass energy of $\sqrt{s} = \SI{8}{\tera \electronvolt}$. \\
Since the cross section of top-quark pair events with an associated photon (\ttgamma events) was already measured by CMS \cite{CMS-PAS-TOP-13-011} this analysis takes this measurement as a base and tries to go beyond it. This is done by either trying to improve the measurement or looking at how the existing result could be used for further analyses. \\
The first four chapters introduce the theoretical framework of the SM, the technical details of the CMS detector and the general simulation and reconstruction strategies used for data and simulation in this analysis. Chapter five shows a new method to simulate \ttgamma events. In order to validate this method two matrix element generators are compared. In chapter six the new simulation strategy is used for a benchmark study investigating a possibly anomalous electric-dipole moment of the top quark. For chapter seven the original CMS \ttgamma measurement is recreated using a new template fit, which is then tested with a closure test. The whole analysis is concluded in chapter eight and an outlook is given. 