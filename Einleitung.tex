\chapter{Introduction}

If there is one sentence that characterizes particle physics it certainly is the question: What are the ingredients of matter? \\
Over the centuries, there have been numerous answers. Today, the best answer is summarized in the Standard Model (SM) of particle physics. The SM has been wildly successful in predicting experimental results during the last years. Its latest success is the discovery of the Higgs boson by the CMS and ATLAS experiments in 2012 \cite{Aad:2012tfa,Chatrchyan:2012ufa} after it had first been predicted in 1964 \cite{Englert:1964et,Higgs:1964ia}. This discovery offers a mechanism for the creation of massive particles. \\
Despite the success of the SM, several fundamental questions remain open: Can the SM's three forces be unified? What is the solution to the Higgs hierarchy problem? What is dark matter? \\
These questions can probably not be answered within the SM framework. Throughout the years, many extensions and alternatives to the SM have been proposed. However, until today no conclusive evidence has been found that any of them is correct. This is the reason for the importance of the Large Hadron Collider (LHC) at Cern, Geneva. It is one of the instruments that allows to probe the SM at previously unknown energy scales. \\
Top quarks belong to the particles commonly produced at the LHC. Moreover, they are objects well suited to testing theories beyond the SM (BSM). Many BSM theories predict stronger differences from the SM for heavy or third generation fermions \cite{AguilarSaavedra:2008zc}. Additionally, top quark couplings can be measured without interference from hadronization. These reasons render the top sector an ideal target for generic searches for BSM physics. \\
The presented analysis focuses on the top-quark pair events with an associated photon (\ttgamma events). It uses data with an integrated luminosity of $\mathcal{L}_{int} = \SI{19.7}{\per \femto \barn}$ taken at the LHC with the CMS detector at a center of mass energy of $\sqrt{s} = \SI{8}{\tera \electronvolt}$. \\
Since the \ttgamma cross section was already measured by CMS \cite{CMS-PAS-TOP-13-011}, the presented analysis considers the CMS measurement as a starting point and extends it. This is done here by both improving the measurement and looking at how the existing result can be used for further analyses. Both aspects pave the way for a better understanding of $\ttgamma$ events. This is especially desirable when the LHC reaches higher energies in the next years. Since $\ttgamma$ events are an important background to a lot of searches for explicit BSM signatures, a well measured top-photon coupling is very important. In order to improve the \ttgamma cross section measurement, the uncertainties are reduced. Since new data is not avaiable at the moment, a focus on the improvement of the systematic uncertainty is feasible.\\
The first four chapters introduce the theoretical framework of the SM, the technical details of the LHC and the CMS detector and the general simulation and reconstruction strategies used for data and simulation in this analysis. Chapter five shows a new method to simulate \ttgamma events. In order to validate this method, two matrix element generators are compared. In chapter six, the new simulation strategy is used for a benchmark study investigating a possibly anomalous magnetic-dipole moment of the top quark. For chapter seven, a \ttgamma cross section measurement is implemented using a new template fit, which is then with a closure test. The whole analysis concludes in chapter eight, followed by an outlook. 